
\documentclass[a4paper,fleqn,usenatbib]{mnras}

\usepackage{newtxtext,newtxmath}
% Depending on your LaTeX fonts installation, you might get better results with one of these:
%\usepackage{mathptmx}
%\usepackage{txfonts}

\usepackage[T1]{fontenc}
\usepackage{ae,aecompl}
\usepackage{booktabs}


\usepackage{graphicx}	% Including figure files
\usepackage{amsmath}	% Advanced maths commands
\usepackage{amssymb}	% Extra maths symbols
\usepackage{color}
\newcommand{\todo}[1]{\textcolor{red}{#1}}
\newcommand{\countertodo}[1]{\textcolor{green}{#1}}

\makeatletter
\newcommand{\rmnum}[1]{\romannumeral #1}
\newcommand{\Rmnum}[1]{\expandafter\@slowromancap\romannumeral #1@}
\makeatother



%%%%%%%%%%%%%%%%%%%%%%%%%%%%%%%%%%%%%%%%%%%%%%%%%%

%%%%% AUTHORS - PLACE YOUR OWN COMMANDS HERE %%%%%

% Please keep new commands to a minimum, and use \newcommand not \def to avoid
% overwriting existing commands. Example:
%\newcommand{\pcm}{\,cm$^{-2}$}	% per cm-squared

%%%%%%%%%%%%%%%%%%%%%%%%%%%%%%%%%%%%%%%%%%%%%%%%%%


% Title of the paper, and the short title which is used in the headers.
% Keep the title short and informative.
\title[Barium stars in LAMOST]{Discovery of barium stars in the LAMOST survey}

% The list of authors, and the short list which is used in the headers.
% If you need two or more lines of authors, add an extra line using \newauthor
\author[B. J. Norfolk et al.]{Brodie J. Norfolk,$^{1}$\thanks{E-mail: bjee7@student.monash.edu (MU)}
Andrew R. Casey,$^{1,2}$
Mathew T. Miles,$^{1}$
Alexander J. Kemp,$^{1}$ \newauthor
Karakas ? Lattanzio ? Schlaufman ? and others ?
\\
$^{1}$School of Physics \& Astronomy, Monash University, Clayton 3800, Victoria, Australia\\
$^{2}$Faculty of Information Technology, Monash University, Clayton 3800, Victoria, Australia\\
}
\date{Accepted XXX. Received YYY; in original form ZZZ}

\pubyear{2018}

\begin{document}
\label{firstpage}
\pagerange{\pageref{firstpage}--\pageref{lastpage}}
\maketitle

\begin{abstract}
Stars with peculiar enhancements of carbon and heavy elements ($Z > 30$) are historically referred to as Barium stars. While this chemical abundance pattern can be explained by thermally pulsing asymptotic giant branch (AGB) stars, many barium stars have not evolved through the red giant (or main-sequence) phase and therefore cannot be an AGB star. These stars are generally thought to have accreted mass from a more evolved companion. For this reason the frequency and properties of less-evolved barium stars is uniquely informative on the binary fraction, AGB yields, mass transfer, and stellar evolution. Here we present the discovery of 257 barium stars from 454,180 giant stars observed by LAMOST. This sample constitutes the largest sample of barium stars discovered to date, and the most comprehensive search for barium stars in the Milky Way. \todo{Nearly all} of our barium stars are disk stars, and we find an \todo{steady/increasing/decreasing} fraction of barium stars with stellar metallicity, as \todo{expected} from AGB yields and stellar evolution theory.  
% Technitium?
\todo{Most (178/257) of our sample show strong carbon enhancement.}
Despite previous claims indicating that barium stars are preferentially enhanced in sodium due to the NeNa nucleosynthesis cycle occurring in \todo{massive} AGB stars, only 5 ($<2$\,\%) of our Barium stars show any evidence of enhancement in sodium. \todo{This implies ....}
 %Barium stars exist as either intrinsic TP-AGB stars, producing enhancements in s-process elements via cyclic thermal pulses. Or, extrinsic giants still on the main sequence with enhancements in s-process element the results of polluton from a binary TP-AGB star. Enhancements in \ion{Ba}{II} and \ion{Sr}{II} resonance lines are very strong, allowing us to identify stars enhanced in neutron-capture elements through a data-driven analysis given a well-fitted model. 
% We determined ... Sodium enrichment was present in 5 of our barium star candidates and could be attributed to the NeNa cycle in more evolved stars.
\end{abstract}

\begin{keywords}
keyword1 -- keyword2 -- keyword3
\end{keywords}


% AGB stars and the s-process.

% Comparisons to expectations of theoretical birth rate
% Binary fraction at different metallicities
% Mass differentials of binary stars
% Yields of AGB stars at different masses and metallicities.

\section{Introduction} \label{sec:intro}

Barium stars as first recognized by \citet{Bidelman1951}, are chemically unique objects that exhibit envelopes with an overabundance of both carbon, and heavy elements (Z > 30) in comparison to the Sun. Barium stars exist as either intrinsic or extrinsic objects; intrinsic barium stars are in the TP-AGB (thermally pulsing-asymptotic giant branch) phase, the overabundance of both carbon and heavy elements are produced through the s-process and exist on the surface as the result of a thermal pulsing cycle. According to the mass-transfer hypothesis, extrinsic Barium stars are a consequence of stellar wind accretion \citep{boffin1988,jorissen1992} or Roche-lobe overflow \citep{webbink1986}, and are within a binary system containing a previous TP-AGB companion star (intrinsic Barium star) in its final phase as a white dwarf \citep{bohm1980,bohm1984}. In fact, \citet{mcclure1983} determines 85\% of all Barium stars are in binary systems and those that appear singular are actually pole-on, or highly eccentric binaries with significant radial velocity variations only occuring in a small phase range as detailed by \citet{pourbaix2004}. For these reasons, the properties and occurrence rate of barium stars are informative of the theoretical birth rate of AGB stars, the binary star fraction as a function of metallicity, the mass ratio of binary stars, as well as AGB yields across different masses and metallicities. 

The slow neutron capture process (s-process), occuring in the interior of AGB stars, synthesizes roughly half of all elements heavier than iron \citep[e.g.][]{busso1999,travaglio2001,herwig2005,romano2010,kobayashi2011,prantzos2012,bisterzo2014,karakas12016}. Late in the AGB phase, during the thermally pulsing-AGB (TP-AGB) phase, thermal instabilities occur in the He shell every $10^5$ years or so, depending on the H-exhausted core mass. These energy bursts drive a convective zone that sweeps the entire region lying between the core mass and He-shell; mixing the products of nucleosynthesis within these regions. The energy from the thermal pulses force the star to expand, pushing the H-shell out to cooler regions and allowing the convective envelope to move inwards to regions previously mixed by the thermal pulse driven convective zones. This expansion and resultsing inward movement is classified as the third dredge up (TDU), and is theorized to occur after each thermal pulse. During the TP-AGB phase, enrichent on the surface in $^{12}$C and heavy elements produced by the s-process is a result of the repeated downwash extensions of TDU \citep[e.g.][]{busso2001}. The star contracts PDU and reignites the H-shell producing the majority of the surface luminosity for the next interpulse period. The interpulse, thermal pulse, and dredge up cycle may occur numerous times and is dependent on the intitial mass, composition and mass-loss rate of the star.

Barium giants form within a metallicity dependent initial mass range (approximately 0.8 - 8$M_{\odot}$), with the minimum mass for core helium and carbon burning decreasing as metalicity decreases. The age of these stars vary considerably, with observable stars in metal-poor globular clusters hosting populations of $\approx$12Gyr. In contrast, metal rich, young observable stars reaching the upper limit may only reach $\approx$100Myr, this includes stars that are at the core carbon burning limit or very close to it \citep[e.g.][]{whitelock2013}.

Many barium stars have been discovered in the disk and halo of the Milky Way \citep{gomez1997,mennessier1997}. Both studies show that the stellar populations of Barium stars can be separated into two groups when considering their \todo{absolute?} luminosity, kinematic, and spatial properties. Barium stars populating the halo have already been investigated \citep[e.g.][]{junqueira2001,drake2008,pereira2009,allen2006} and their relationship with metal-poor, yellow, symbiotic halo stars has been established by \citet{jorissen2005} and \citet{pereira2009}. Additionally, \citet{pereira2011} concludes metal-rich barium stars share similar kinematics to other metal-rich and super metal-rich stars already analysed, suggesting that they do not belong to the bulge population. Furthermore, it is necessary to establish known abundances of s-process elements in a greater sample of Barium stars in order to compare their birth rate to theoretical expectations \citep{han1995}, explore the binary fraction at different metallicities, and infer AGB star yields at different masses and metallicities.

In this paper we analysis 454,180 giant stars from LAMOST data release (DR) 2 to identify and examine 257 Barium star candidates that show enhancements of s-process elements. Specifcally, the giants studied here display especially strong s-process enhancement for \ion{Ba}{II} and \ion{Sr}{II} at lines 4554\AA \hspace{0.2mm} and 4077\AA, respectively. Additionally, molecular band absorption for G Band, CH bands is observed. In Section \ref{sec:methods} we describe the observations and candidate selection, as well as the analysis follow up our follow-up observing campaign with MAGGELLAN. In Section \ref{sec:dis} we discuss the properties of our Barium stars in context of existing literature. We provide concluding remarks in Section \ref{sec:con}

\section{Methods} \label{sec:methods}
\subsection{LAMOST Analysis}
\subsubsection{Data-driven Analysis}
The spectral data of our stars were obtained from the \textit{Cannon} fits predictive model implemented by \citet{ho2017} on the LAMOST (Large Sky Area Multi-Object Fibre Spectroscopic Telescope) DR2 release \citep{luo2015}. The LAMOST spectral resolving power of R = 1800, a wavelength coverage from 3700\AA\ to 9000\AA, and \todo{a nominal S/N ratio of 100 (is this true?)}\countertodo{I read it in a paper somewhere, is it not true ?}. Stellar parameters ($T_{\rm eff}$, $\log{g}$, [M/H], [$\alpha$/M]) were derived using a data-driven model constructed using high fidelity APOGEE labels of 9,952 stars in common between LAMOST and the APOGEE survey.

\subsubsection{Candidate Selection} \label{sec:cand}
The Barium stars analysed in this research were obtained by filtering for significant flux residuals, taken as, a disparity between the normalised LAMOST flux and the data driven model derived by \citet{ho2017}. A negative residual as shown in the poster child figure \ref{fig:figure1} illustrates an enhancement in \ion{Ba}{II} and \ion{Sr}{II} at lines 4554\AA \hspace{0.2mm} and 4077\AA, respectively. These \ion{Ba}{II} and \ion{Sr}{II} resonance lines are very strong, allowing us to identify stars enhanced in neutron-capture elements given a well-fit model for the data. We used five filters that each spectra, for both enhancement lines, had to meet to be considered a Barium star candidate; these included:

\renewcommand\labelenumi{(\roman{enumi})}
\renewcommand\theenumi\labelenumi

\begin{enumerate} 
\item Profile amplitude for both enhancement lines must exceed A$\leq$-0.05.
\item Both amplittudes must be measured within 3$\sigma$ $(|A|/\sigma _A \geq 3)$.
\item The wavelength at each absorption line must be within $2$\AA \hspace{0.2mm}$(\lambda \leq 2)$.
\item $\chi^2 \leq 3$.
\item And, the spectra must have a signal-to-noise ratio of S/N $\geq 30$.
\end{enumerate}
In addition, a visual observation of the accuracy between each normalised LAMOST flux and its accompanying data driven model was undertaken to exclude any results containing, false positives, candidates exhibiting data reduction issues, apparent absorption finer then the spectral resolution, and overly noisy normalised LAMOST spectra. This combined method generated the 257 Barium star candidates as listed in Table \ref{table:table1}. 

\begin{figure}
	% To include a figure from a file named example.*
	% Allowable file formats are eps or ps if compiling using latex
	% or pdf, png, jpg if compiling using pdflatex
	\includegraphics[width=\columnwidth]{posterchild.png}
    \caption{Posterchild}
    \label{fig:figure1}
\end{figure}

\begin{table}[]
\centering
\caption{Available online in its entirety. Here we show a portion to demonstrate its style and content}
\label{table:table1}
\begin{tabular}{@{}|l|l|l|l|l|l|l|l|l|l|@{}}
\toprule
2MASSID             & RA (hms         & DEC (hms)        & Vr (km/s) & S/N & Teff (K) & Log g & {[}Fe/H{]} & {[}$\alpha$/H{]} & $\chi^2$  \\ \midrule
J000134.95+490743.2 & 00:01:34.96 & +49:07:43.2 & -42.27    & 72  & 5044 & 3.11  & -0.54      & 0.11      & 0.79 \\ \midrule
J000403.80+160257.1 & 00:04:03.80 & +16:02:57.2 & -35.38    & 42  & 5200 & 3.40  & -0.41      & 0.09      & 0.33 \\ \midrule
J000908.48+421615.4 & 00:09:08.49 & +42:16:15.5 & -34.78    & 36  & 4825 & 3.02  & -0.46      & 0.16      & 0.25 \\ \midrule
J001754.17+394140.0 & 00:17:54.17 & +39:41:40.0 & -42.87    & 39  & 5008 & 3.16  & -0.64      & 0.12      & 0.23 \\ \midrule
J001841.57+040533.7 & 00:18:41.57 & +04:05:33.8 & -20.69    & 141 & 4840 & 3.68  & -0.52      & 0.06      & 0.93 \\ \midrule
J002107.60+361533.0 & 00:21:07.60 & +36:15:33.1 & -53.06    & 44  & 4854 & 2.87  & 0.01       & 0.06      & 0.54 \\ \midrule
J002537.59+411123.7 & 00:25:37.59 & +41:11:23.8 & -12.59    & 36  & 5075 & 3.30  & -0.32      & 0.10      & 0.33 \\ \midrule
J003755.55+571706.8 & 00:37:55.55 & +57:17:06.9 & -5.70     & 49  & 4046 & 1.14  & -0.09      & -0.01     & 0.83 \\ \midrule
J003813.23+423025.9 & 00:38:13.23 & +42:30:25.9 & -40.17    & 36  & 4475 & 2.12  & 0.14       & 0.00      & 0.51 \\ \midrule
J003959.14+394614.6 & 00:39:59.14 & +39:46:14.6 & 5.10      & 35  & 4955 & 3.23  & -0.40      & 0.11      & 0.28 \\ \midrule
J004119.68+324627.8 & 00:41:19.68 & +32:46:27.9 & 4.50      & 122 & 4861 & 2.59  & -0.31      & 0.08      & 1.67 \\ \midrule
J004156.56+390945.6 & 00:41:56.57 & +39:09:45.6 & -254.52   & 44  & 4771 & 2.34  & -0.74      & 0.16      & 0.57 \\ \midrule
J004743.01+433358.1 & 00:47:43.02 & +43:33:58.2 & -62.66    & 65  & 4834 & 2.64  & -0.60      & 0.17      & 1.06 \\ \midrule
J005101.98+314045.6 & 00:51:01.98 & +31:40:45.7 & -23.68    & 98  & 5202 & 3.32  & -0.53      & 0.10      & 0.67 \\ \midrule
J005254.69+203554.7 & 00:52:54.69 & +20:35:54.7 & 35.68     & 113 & 5026 & 3.17  & -0.82      & 0.20      & 1.01 \\ \midrule
J010836.26+444435.9 & 01:08:36.26 & +44:44:35.9 & 12.59     & 49  & 5189 & 3.19  & -0.56      & 0.10      & 0.98 \\ \midrule
J011919.20+195227.0 & 01:19:19.21 & +19:52:27.0 & -35.38    & 31  & 4897 & 2.77  & -0.54      & 0.11      & 0.29 \\ \midrule
J011930.90-015027.3 & 01:19:30.91 & -01:50:27.4 & -143.30   & 44  & 5040 & 3.22  & -0.91      & 0.20      & 0.48 \\ \midrule
J020316.40+462211.1 & 02:03:16.41 & +46:22:11.1 & 11.09     & 37  & 4897 & 2.89  & -0.39      & 0.11      & 0.51 \\ \bottomrule
\end{tabular}
\end{table}

\subsubsection{Technetium Lines}

Enhancement of technetium lines in our sample would indicate the presence of high luminosity Tc-self-enriched AGB S stars \citep{jorissen1993} and, suggest our candidates are infact intrinsic Barium stars. We undertook an identical analysis to the process described in Section \ref{sec:cand}, negative residuals for Tc enhancement lines 4049\AA, 4238.19\AA, 4262.27\AA, 4297\AA, and 5924.47\AA\hspace{0.2mm} were filtered for. In agreement with previous findings \citep[e.g.][]{little1987,smith1984,smith1983}, the results indicate that none of the evolved barium stars in our sample exhibited the presence of Tc enhancement. It is noted, \todo{51} of our barium candidates were enhanced in Tc lines, although, this is attributed to \todo{something} and is disregarded.

\subsubsection{Carbon Bands}

Barium stars are optimal sites for investigating the relationship between the neutron-capture elements and other species that may be
depleted or enhanced, because they act as neutron seeds during the operation of the s-process. In the ABG phase the abundance of carbon is a product of helium fusion, specifically the triple-alpha process within a star. In an identical mechanism to the presence of s-process elelment enhancement, carbon enhanced stars exist intrinsically or extrinsically. Giants and supergiants become enhanced in carbon through TDU process discussed in Section \ref{sec:intro}, and extrinsic carbon enhancement occurs via mass transfer from a binary AGB star. In our sample of barium star candidates we discovered enhancement in both CH and the G band for 178 of our total 257 stars. \todo{maybe put a list of the stars ?}

\subsubsection{Sodium Doublet}

According to \citet{el1995}, sodium overabundance observed in the atmospheres of A-F supergiant stars is synthesized in the convective core of main-sequence stars in the NeNa reaction chain. Sodium is transported to the surface of these supergiants via mixing of CNO cycle products during the first dredge-up, this constrains sodium enhancement to supergiants and giants only. \citet{antipova2004} discovered a sodium enhancement in a number of their barium star candidates, they relate this to the dredge-up of nuclear-burning material produced by convection during the red-giant phase. Moreover, they suggest [Na/Fe] values do not exhibit variations in metallicity, but [Na/Fe] ratios exhibit systematically higher values for giants with lower log g values. Similarily, \citet{decastro2016} highlights the possible weak, but statistically significant anti-correlation between [Na/Fe] ratio and log g, and illustrates this trend in previous studies\citep[e.g.][]{boyarchuk2002,mishenina2006,luck2007,takeda2008}. 

Through an identical analysis to the process described in Section \ref{sec:cand}, we filtered for negative residuals in the enhanced sodium doublet lines 5889.950\AA\hspace{0.2mm} and 5895.924\AA. Our analysis produced 5 matches with our barium star candidates, this suggests sodium-rich material from the TP-AGB star polluted the barium star. Although, it's important to note a star with a minimum 1.5$M_\odot$ can exhibit sodium abundance through the NeNc cycle \citep{denissenkov1987}. During the giant phase synthesized sodium is brought to the surface via the first dredge-up, therefore, these candidates may display sodium enhancement without mass transfer from a TP-AGB star.
\todo{maybe put a list of the stars ?}

\subsection{MAGELLAN Analysis}

\subsubsection{Observations}

\subsubsection{Abundance Analysis}

\subsubsection{Abundance Uncertainties}

\subsection{Dynamics}
The dynamical analysis of our barium star candidates was accomplished by crossing matching the sample with GAIA \todo{ref gaia}, producing 21 matches, and then using gala Python package to intergrate the galactic dynamics. The Milky Way potential model was used to integrate the orbits of our candidates back a \todo{number} years, and compute distributions of orbital properties including orbital period, orbital energy, apocenter, eccentricity, and angular momentum in z.

\begin{figure}
	% To include a figure from a file named example.*
	% Allowable file formats are eps or ps if compiling using latex
	% or pdf, png, jpg if compiling using pdflatex
	\includegraphics[width=\columnwidth]{ECCdummy.png}
    \caption{Eccentricity}
    \label{fig:figure3}
\end{figure}

\begin{figure}
	% To include a figure from a file named example.*
	% Allowable file formats are eps or ps if compiling using latex
	% or pdf, png, jpg if compiling using pdflatex
	\includegraphics[width=\columnwidth]{energy_angmomz.png}
    \caption{Energy Vs. Angular Momentum in Z}
    \label{fig:figure3}
\end{figure}

\section{Discussion}  \label{sec:dis}

\subsection{HR Diagram}
\begin{figure}
	% To include a figure from a file named example.*
	% Allowable file formats are eps or ps if compiling using latex
	% or pdf, png, jpg if compiling using pdflatex
	\includegraphics[width=\columnwidth]{HRdummy.png}
    \caption{HR Diagram}
    \label{fig:figure2}
\end{figure}


\section{Conclusions} \label{sec:con}

The last numbered section should briefly summarise what has been done, and describe
the final conclusions which the authors draw from their work.

\section*{Acknowledgements}

Fired Status: yes

%\section{Examples}
%
%Normally the next section describes the techniques the authors used.
%It is frequently split into subsections, such as Section~\ref{sec:maths} below.
%
%\subsection{Maths}
%\label{sec:maths} 
%
%Simple mathematics can be inserted into the flow of the text e.g. $2\times3=6$
%or $v=220$\,km\,s$^{-1}$, but more complicated expressions should be entered
%as a numbered equation:
%
%\begin{equation}
%    x=\frac{-b\pm\sqrt{b^2-4ac}}{2a}.
%	\label{eq:quadratic}
%\end{equation}
%
%Refer back to them as e.g. equation~(\ref{eq:quadratic}).
%
%\subsection{Figures and tables}
%
%Figures and tables should be placed at logical positions in the text. Don't
%worry about the exact layout, which will be handled by the publishers.
%
%Figures are referred to as e.g. Fig.~\ref{fig:example_figure}, and tables as
%e.g. Table~\ref{tab:example_table}.
%
%\begin{figure}
%	% To include a figure from a file named example.*
%	% Allowable file formats are eps or ps if compiling using latex
%	% or pdf, png, jpg if compiling using pdflatex
%	\includegraphics[width=\columnwidth]{example}
%    \caption{This is an example figure. Captions appear below each figure.
%	Give enough detail for the reader to understand what they're looking at,
%	but leave detailed discussion to the main body of the text.}
%    \label{fig:example_figure}
%\end{figure}
%
%
%\begin{table}
%	\centering
%	\caption{This is an example table. Captions appear above each table.
%	Remember to define the quantities, symbols and units used.}
%	\label{tab:example_table}
%	\begin{tabular}{lccr} % four columns, alignment for each
%		\hline
%		A & B & C & D\\
%		\hline
%		1 & 2 & 3 & 4\\
%		2 & 4 & 6 & 8\\
%		3 & 5 & 7 & 9\\
%		\hline
%	\end{tabular}
%\end{table}


% The best way to enter references is to use BibTeX:
 
\bibliographystyle{mnras}
\bibliography{example} % if your bibtex file is called example.bib

\begin{thebibliography}{99}
\bibitem[\protect\citeauthoryear{Author}{2012}]{Author2012}
Author A.~N., 2013, Journal of Improbable Astronomy, 1, 1
\bibitem[\protect\citeauthoryear{Others}{2013}]{Others2013}
Others S., 2012, Journal of Interesting Stuff, 17, 198


\bibitem[\protect\citeauthoryear{Bidelman \& Keenan}{1951}]{Bidelman1951}
Bidelman, W.~P. \& Keenan, P.~C. , 1951, ApJ, 114, 473
\bibitem[\protect\citeauthoryear{Busso et al.}{1999}]{busso1999}
Busso, M., Gallino, R., \& Wasserburg, P.~C. 1999, 
ARA$\&$A, 37, 239
\bibitem[\protect\citeauthoryear{Travaglio et al.}{2001}]{travaglio2001}
Travaglio, C., et al. 2001, 
ApJ, 549, 346
\bibitem[\protect\citeauthoryear{Herwig}{2005}]{herwig2005}
Herwig, F. 2005, 
ARA$\&$A, 43, 435
\bibitem[\protect\citeauthoryear{Romano et al.}{2010}]{romano2010}
Romano, D., et al. 2010, 
A$\&$A, 522, A32
\bibitem[\protect\citeauthoryear{Kobayashi et al.}{2011}]{kobayashi2011}
Kobayashi, C., Karakas, A., \& Umeda, H. 2011, 
MNRAS, 414, 3231
\bibitem[\protect\citeauthoryear{Prantzos}{2012}]{prantzos2012}
Prantzos, N. 2012, 
A$\&$A, 542, A67
\bibitem[\protect\citeauthoryear{Bisterzo et al.}{2014}]{bisterzo2014}
Bisterzo, S., et al. 2014, 
ApJ, 787, 10
\bibitem[\protect\citeauthoryear{Karakas}{2016}]{karakas12016}
Karakas, A. 2016, 
SAIt, 87, 229
\bibitem[\protect\citeauthoryear{Busso et al.}{2001}]{busso2001}
Busso, M., Gallino, R., Lambert, D.~L., Travaglio, C.\& Smith, V.~V. 2001, 
ApJ, 557, 802
\bibitem[\protect\citeauthoryear{McClure}{1983}]{mcclure1983}
McClure, R.~D. 1983, 
ApJ, 268, 264
\bibitem[\protect\citeauthoryear{Pourbaix et al.}{2004}]{pourbaix2004}
Pourbaix, D., Tokovinin, A.~A., Batten, A.~H., Fekel, F.~C., Hartkopf, W.~I. et al. 2001, 
A$\&$A, 424, 727
\bibitem[\protect\citeauthoryear{B\"ohm-Vitense}{1980}]{bohm1980}
B\"ohm-Vitense, E. 1980, 
ApJ, 239, L79
\bibitem[\protect\citeauthoryear{B\"ohm-Vitense et al.}{1984}]{bohm1984}
B\"ohm-Vitense, E., Nemec, J.,\& Proffitt, C. 1984, 
ApJ, 278, 726
\bibitem[\protect\citeauthoryear{Boffin \& Jorissen}{1988}]{boffin1988}
Boffin H, M.~J.,\& Jorissen, A. 1988, 
A$\&$A, 205, 155
\bibitem[\protect\citeauthoryear{Jorissen \& Boffin}{1992}]{jorissen1992}
Jorissen, A.,\& Boffin H, M.~J., 1992, 
Evidences for interaction among wide binary systems: To Ba or not to Ba? In: Duquennoy, A., Mayor, M.,(eds.) Binaries as tracers of stellar formation. Cambridge Univ. Press., p.185
\bibitem[\protect\citeauthoryear{Webbink}{1986}]{webbink1986}
Webbink, R.~F. 1986, 
In: Leung, K.~C., Zhai, D.~S.(eds.) Critical Observations versus Physical Models for Close Binary Systems. Gordon and Breach, New York, p.403
\bibitem[\protect\citeauthoryear{Whitelock et al.}{2013}]{whitelock2013}
Whitelock, P.~A., et al. 2013, 
MNRAS, 428, 2216
\bibitem[\protect\citeauthoryear{Gomez et al.}{1997}]{gomez1997}
Gomez, A.~E., Luri, X., Grenier, S., et al. 1997, 
A$\&$A, 319, 881
\bibitem[\protect\citeauthoryear{Mennessier et al.}{1997}]{mennessier1997}
Mennessier, M.~O., Luri, X., Figueras, F., et al. 1997, 
A$\&$A, 326, 722
\bibitem[\protect\citeauthoryear{Junqueira \& Pereira}{2001}]{junqueira2001}
Junqueira S.,\& Pereira, C.~B. 2001, 
AJ, 122, 360
\bibitem[\protect\citeauthoryear{Drake \& Pereira}{2008}]{drake2008}
Drake N.,~A.,\& Pereira, C.~B. 2008, 
AJ, 135, 1070
\bibitem[\protect\citeauthoryear{Pereira \& Drake}{2009}]{pereira2009}
Pereira, C.~B.,\& Drake N.,~A. 2009, 
A$\&$A, 496, 791
\bibitem[\protect\citeauthoryear{Allen \& Barbuy}{2006}]{allen2006}
Allen, D.~M.,\& Barbuy B. 2006, 
A$\&$A, 454, 917
\bibitem[\protect\citeauthoryear{Jorissen et al.}{2005}]{jorissen2005}
Jorissen, A., Za$\check{c}$, L., Udry, S., Lindgren, H.,\& Musaev, F.~A. 2005, 
A$\&$A, 441, 1135
\bibitem[\protect\citeauthoryear{Han et al.}{1995}]{han1995}
Han, Z., Eggleton, P.~P., Podsiadlowski, P.,\& Tout, C.~A. 1995, 
MNRAS, 277, 1443
\bibitem[\protect\citeauthoryear{Pereira et al.}{2011}]{pereira2011}
Pereira, C.~B., Sales Silva, J.,~A., Chavero, C., Roig, F.,\& Jilinski E. 2011, 
A$\&$A, 533, A51
\bibitem[\protect\citeauthoryear{Ho et al.}{2017}]{ho2017}
Ho, A.~Y.~Q., Ness, M.~K., Hogg, D.~W., Rix, H.-W. Liu, C., Yang, F., Zhang, Y., Hou, Y.,\& Wang, Y. 2017, 
ApJ, 836, 5
\bibitem[\protect\citeauthoryear{Luo A.L., Bai Z.R. et al}{2015}]{luo2015}
Luo, A.~L., Bai, Z.~R., et al. 2015, 
RAA, in press
\bibitem[\protect\citeauthoryear{Jorissen et al.}{1993}]{jorissen1993}
Jorissen, A., Frayer, D.~T., Johnson, H.~W., Mayor, M.,\& Smith, V.~V. 1993, 
A$\&$A, 271, 463
\bibitem[\protect\citeauthoryear{Little-Marenin \& Little}{1987}]{little1987}
Little-Marenin, I.~R.,\& Little, S.~J. 1987, 
AJ, 93, 1539
\bibitem[\protect\citeauthoryear{Smith}{1984}]{smith1984}
Smith, V.~V. 1984, 
A$\&$A, 132, 326
\bibitem[\protect\citeauthoryear{Smith \& Wallerstein}{1983}]{smith1983}
Smith, V.~V.,\& Wallerstein, G. 1983, 
ApJ, 273, 742
\bibitem[\protect\citeauthoryear{El Eid \& Champagne}{1995}]{el1995}
El Eid, M.~F.,\& Champagne, A.~E. 1995, 
ApJ, 451, 298
\bibitem[\protect\citeauthoryear{Antipova et al.}{2004}]{antipova2004}
Antipova, L.~I., Boyarchuk, A.~A., Pakhomov, Yu.~V.,\& Panchuk, V.~E. 2004, 
ARep, 48, 597
\bibitem[\protect\citeauthoryear{Boyarchuk et al.}{2002}]{boyarchuk2002}
Boyarchuk, A.~A., Pakhomov, Y.~V., Antipova, L.~I.,\& Boyarchuk, M.~E. 2002, 
ARep, 46, 819
\bibitem[\protect\citeauthoryear{Takeda et al.}{2008}]{takeda2008}
Takeda, Y., Sato, B.~V.,\& Murata, D. 2008, 
AJ, 60, 781
\bibitem[\protect\citeauthoryear{Mishenina et al.}{2006}]{mishenina2006}
Mishenina, T.~V., Bienaym\' e, O., Gorbaneva, T.~I.,\& Charbonnel, C. 2006, 
A$\&$A, 456, 1109
\bibitem[\protect\citeauthoryear{Luck \& Heiter}{2007}]{luck2007}
Luck R.~E.,\& Heiter, U. 2007, 
AJ, 133, 2464
\bibitem[\protect\citeauthoryear{Denissenkov \& Ivanov}{1987}]{denissenkov1987}
Denissenkov P.~A.,\& Ivanov, V.~V. 1987, 
SvAL, 13, 214
\bibitem[\protect\citeauthoryear{de Castro et al.}{2016}]{decastro2016}
de Castro, D.~B., Pereira, C.~B., Roig, F., Jilinski, E., Drake, N.~A., Chavero, C.,\& Sales Silva, J.~V. 2016, 
MNRAS, 459, 4299


\end{thebibliography}


\appendix

\section{Some extra material}

If you want to present additional material which would interrupt the flow of the main paper,
it can be placed in an Appendix which appears after the list of references.

% Don't change these lines
\bsp	% typesetting comment
\label{lastpage}
\end{document}