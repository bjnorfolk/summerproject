
\documentclass[a4paper,fleqn,usenatbib]{mnras}

\usepackage{newtxtext,newtxmath}
% Depending on your LaTeX fonts installation, you might get better results with one of these:
%\usepackage{mathptmx}
%\usepackage{txfonts}

\usepackage[T1]{fontenc}
\usepackage{ae,aecompl}
\usepackage{booktabs}


\usepackage{graphicx}	% Including figure files
\usepackage{amsmath}	% Advanced maths commands
\usepackage{amssymb}	% Extra maths symbols
\usepackage{color}

%%%%%%%%%%%%%%%%%%%%%%%%%%%%%%%%%%%%%%%%%%%%%%%%%%

%%%%% AUTHORS - PLACE YOUR OWN COMMANDS HERE %%%%%
\usepackage{float}
\newcommand{\todo}[1]{\textcolor{red}{#1}}
\newcommand{\countertodo}[1]{\textcolor{green}{#1}}
\usepackage{adjustbox}
\usepackage{pifont}




\makeatletter
\newcommand{\rmnum}[1]{\romannumeral #1}
\newcommand{\Rmnum}[1]{\expandafter\@slowromancap\romannumeral #1@}
\makeatother

\title[S-process stars in LAMOST]{Discovery of s-process enhanced stars in the LAMOST survey}

\author[Brodie.~J. Norfolk et al.]{Brodie.~J. Norfolk,$^{1}$\thanks{E-mail: bjee7@student.monash.edu (MU)}\thanks{This paper includes data gathered with the 6.5 meter Magellan Telescopes located at Las Campanas Observatory, Chile.}
Andrew.~R. Casey$^{1,2}$,
Matthew ~T. Miles$^{1}$,
Alex ~J. Kemp$^{1}$, \newauthor
Amanda ~I. Karakas$^{1}$,
Kevin ~C. Schlaufman$^{4}$,
Melissa Ness$^{5}$,
Anna Y.~Q. Ho$^{3}$, \newauthor
John ~C. Lattanzio$^{1}$, 
Alexander ~P. Ji$^{6}$
\\
$^{1}$School of Physics \& Astronomy, Monash University, Clayton 3800, Victoria, Australia\\
$^{2}$Faculty of Information Technology, Monash University, Clayton 3800, Victoria, Australia\\
$^{3}$Cahill Center for Astrophysics, California Institute of Technology, MC 249-17, 1200 E California Blvd, Pasadena, Ca, 91125, USA\\
$^{4}$Department of Physics and Astronomy, Johns Hopkins University, 3400 N Charles St., Baltimore, MD 21218, USA
\\
$^{5}$Department of Astronomy, Columbia University, 550 West 120th Street New York, New York 10027
\\
$^{6}$Carnegie Observatories
}
\date{Accepted XXX. Received YYY; in original form ZZZ}

\pubyear{2018}


\begin{document}
\label{firstpage}
\pagerange{\pageref{firstpage}--\pageref{lastpage}}
\maketitle

\begin{abstract}

Here we present the discovery of 895 stars enriched in s-process elements from the LAMOST spectra using a data-driven approach. This sample constitutes the largest number of s-process enhanced stars ever discovered. Our sample includes; 187 s-process candidates that are enhanced in both barium and strontium, 49 stars with significant barium enhancement only, and 659 stars that show only strontium enhancement. Our entire s-process candidate sample exists within temperature and luminosity ranges not enveloped by the AGB?, which may indicate a recent ($<1\,\textrm{Myr}$) mass accretion event from an AGB companion. The majority of our barium stars ($95\%$) show strong carbon enhancements yet, only 5 candidates ($<3$\,\%) show evidence of sodium enhancement, in tension with previous observations of enhanced Na among s-process candidates. We argue this may have been a selection bias? in previous studies. Our kinematical analysis reveals our sample is composed of 97\% disk stars, and 3\% with velocity consistent to the halo. A comparison with AGB yields suggests the main neutron source responsible for Ba and Sr enhancements is the $\text{C}^{13}$(\textrm{$\alpha$},n)$\text{O}^{16}$ reaction chain. Theoretical models show that this source is dominant in low-mass AGB stars. We conclude that the progenitors for s-process enhancement in our sample are AGB companion stars with masses $1 - 3\,M_{\odot}$. 

\end{abstract}

\begin{keywords}
stars: chemically peculiar -- stars: abundances
\end{keywords}

\section{Introduction} \label{sec:intro}

The abundance of elements created through the slow neutron-capture process (s-process), is a measure of the nucleosynthetic reactions that occur primarily in Asymptotic Giant Branch (AGB) stars. Due to their associated enrichment of the s-process element barium, stars with peculiar enhancements of carbon, and heavy elements (Z > 30) are commonly referred to as 'barium stars' \citep{Bidelman1951}. The naming convention associated with 'barium stars' is historical, for the purposes of this paper it is necessary to classify definitions used for subsequent discussions. Stars with Ba and Sr enrichment will be refereed to as s-process candidates, and singular enhancement samples will be named according to their respective element. Metallicity-dependent classifications also exist in the literature. Metal-poor stars ([Fe/H] $\approx$ [0,-1.2]) enriched by s-process elements are classically refereed to as CH stars, we will simply define them as metal poor s-process candidates. An abundance criteria may also be used to define s-process enrichment, this is typically taken as threshold of $\rm[Ba/Fe]\geq +0.3$ \citep{malaney1988}. For the purposes of this paper, s-process enrichment will be determined solely on the presence significant flux residuals as described in Section \ref{sec:cand}.
S-process candidates can be classified as either extrinsic or intrinsic objects. Extrinsic s-process stars are stellar phenomena exhibiting an enhancement as the result of some polluting process. Stars intrinsically enhanced in s-process elements are known as AGB S stars. These stars are in the TP-AGB (thermally pulsing-asymptotic giant branch) phase. During this stage of nucleosynthesis, the s-process produces an over-abundance of both carbon and heavy elements. The thermal pulsing cycle transports these elements to surface of TP-AGB stars. According to the mass-transfer hypothesis, extrinsic s-process stars are a consequence of stellar wind accretion \citep{boffin1988,jorissen1992} or Roche-lobe overflow \citep{webbink1986}. Both dynamical systems require a binary configuration with a previous TP-AGB companion star (AGB S star) in its final phase as a white dwarf \citep{bohm1980,bohm1984}. In fact, \citet{mcclure1983} determined 85\% of all s-process candidates are in binary systems. They further claimed that stars which appeared to be singular are actually pole-on, or highly eccentric binaries with significant radial velocity variations only occurring in a small ranges \citep{pourbaix2004}. For these reasons, the properties and occurrence rate of s-process candidates are informative of the binary star fraction as a function of metallicity, the mass ratio of binary stars, as well as AGB yields across different masses and metallicities. 

\begin{figure*}
	% To include a figure from a file named example.*
	% Allowable file formats are eps or ps if compiling using latex
	% or pdf, png, jpg if compiling using pdflatex
	\includegraphics[width=\textwidth]{posterchild_final.pdf}
	\caption{Pseudo-continuum-normalised LAMOST spectra for the s-process candidate J064451.11+263239.2. The data are shown in black and the best-fitting data-driven model is shown in red. We include zoom-in axes to show significant deviations in Sr and Ba at  4077\,\AA\ and 4554\,\AA\, respectively.}
	\label{fig:figure1}
\end{figure*}

Occurring in the interior of AGB stars, the s-process synthesises roughly half of all elements heavier than iron \citep[e.g.,][]{busso1999,travaglio2001,herwig2005,bisterzo2014,karakas2014}. During the subsequent TP-AGB phase, thermal instabilities occur in the He shell every $10^5$ years or so, depending on the mass of the H-exhausted core. These energy bursts drive a convective zone that sweeps almost the entire region lying between the core and the He-shell, mixing the products of nucleosynthesis within these regions. While simultaneously, forcing the star to expand and pushing the H-shell out to cooler regions \citep{karakas2002}. The convective envelope will then move inwards towards regions previously mixed by the thermal pulse driven convective zones. This phase of expansion and resulting inward movement is defined as the third dredge up (TDU), and is theorised to occur after each thermal pulse. During the TP-AGB phase, the TDU is responsible for the surface enrichment in $^{12}$C and heavy elements produced by the s-process \citep[e.g.,][]{busso2001}. Post dredge up the star contracts, reigniting the H-shell, and producing the majority of the surface luminosity for the next interpulse period. The interpulse, thermal pulse, and dredge up cycle may occur numerous times and the frequency depends on the initial mass, composition, and mass-loss rate of the star.

S-process candidate can form within a metallicity-dependent initial mass range of approximately $0.8 - 8\,M_{\odot}$ \citep{karakas2016cp}. The minimum mass is defined by the onset of core helium burning, and the maximum by core carbon burning. The age of these stars varies considerably, with some reaching ages $\approx$ 12\,Gyr in metal-poor globular clusters. Less massive metal-rich stars may reach ages of $\approx$ 100Myr, this includes stars that are at the core carbon burning limit or very close to it \citep[e.g.,][]{whitelock2013}. The majority s-process candidates are observed to be members of the disk \citep{jorissen1993,gomez1997,mennessier1997}. In \citet{pereira2011}, it is shown that metal-rich s-process candidates share similar kinematics to other metal-rich and super metal-rich stars already analysed, suggesting that they do not belong to the bulge population. \cite{jorissen1993} relates galactic position to the extrinsic or intrinsic nature of the s-process candidate. They show that intrinsically enhanced stars are more concentrated towards the galactic plane than extrinsic enhanced stars.

In this paper we analyse 454,180 giant stars from the second LAMOST data release \citep{luo2015} and identify 895 s-process candidates. In Section \ref{sec:methods} we describe the observations and candidate selection. Section \ref{sec:observations} details the analysis of high-resolution follow-up observations obtained for a few candidates. In Section \ref{sec:dis} we discuss the properties of our s-process candidates in context of existing literature. We provide concluding remarks in Section \ref{sec:con}.


\section{Methods} \label{sec:methods}
\subsection{LAMOST analysis}
\subsubsection{Data-driven analysis}
The LAMOST (Large sky Area Multi-Object Fibre Spectrographic) survey released low-resolution ($\mathcal{R} \approx 1800$) optical spectra (3700\,\AA\ to 9000\,\AA) for 2,207,189 stars in their second data release \citep{luo2015}. Stellar parameters ($T_{\rm eff}$, $\log{g}$, [M/H], [$\alpha$/M]) were derived by \citep{ho2017} using a \emph{The Cannon} \citet{ness2016}, with labels for 9,952 giant stars in common between LAMOST and the APOGEE survey. We note that this sample is limited to giant stars, implying that we would not identify dwarf s-process enhanced stars. Cross-validation tests shows that the typical uncertainties are approximately 70\,K in effective temperature $\rm T_{\rm eff}$, 0.1\,dex in surface gravity $\log{g}$, 0.1\,dex in metallicity [M/H], and 0.04\,dex in the $\alpha$-element abundance relative to overall metallicity [$\alpha$/M]. These uncertainties are comparable to the those present in APOGEE \citep{alam2015}.


\subsubsection{Candidate selection} \label{sec:cand}
The s-process candidates in this work were identified by filtering for significant flux residuals. We fit a Gaussian distribution with amplitude A to the flux residuals at each s-process element absorption line. These included; 4554\,\AA\ (\ion{Ba}{II}) and 4077\,\AA\ (\ion{Sr}{II}) wavelengths for s-process candidates, 4554\,\AA\ (\ion{Ba}{II}) and 4934\,\AA\ (\ion{Ba}{II}) wavelengths for barium enriched candidates, and wavelengths at 4077\,\AA\ (\ion{Sr}{II}) and 4215\,\AA\ (\ion{Sr}{II}) for strontium enriched candidates. The amplitude of each Gaussian fit at each wavelength is analogous to the depth of each absorption line. The discrepancy is then measured as the difference between the normalised LAMOST flux and the data driven model. A negative residual indicates enrichment, this is shown for a randomly chosen s-process candidate in Figure \ref{fig:figure1}.
We used five filters for each spectrum and subsequent set of absorption lines in order to identify s-process enrichment, these included:

\renewcommand\labelenumi{(\roman{enumi})}
\renewcommand\theenumi\labelenumi

\begin{enumerate} 
\item Profile amplitude for both enhancement lines must be $A < -0.05$, indicating a stronger absorption line than expected by the model.
\item Both amplitudes must be measured within 3$\sigma$ ($|A|/\sigma _A$ < 3).
\item The wavelength at each absorption line must be within 2\,\AA\ ($\lambda$ < 2) of the rest frame laboratory model.
\item The reduced $\chi^2$ from \emph{The Cannon} must be $\chi_r^2 < 3$.
\item The LAMOST spectra must have a signal-to-noise ratio of $S/N > 30\,\textrm{pixel}^{-1}$.
\end{enumerate}
We use visual inspection with all candidates to exclude any results containing false positives: candidates with data reduction issues; apparent absorption finer then the spectral resolution; or overly noisy normalised LAMOST spectra. This approach discovers 895 s-process rich candidates which are provided in Table 1. This sample consists of; 187 s-process candidates, 49 barium enriched stars, and 659 stars with strontium enrichment. It is the largest collection of s-process enhanced stars ever discovered, with the next sample of significant size containing 182 s-process candidates \citep[e.g.,][]{decastro2016}. 

For the remainder of this paper we restrict our analysis to the 187 s-process candidates that show enhancement in both barium and strontium. While the 49 or 659 stars with only barium or strontium enhancements (respectively) may also be classified as s-process candidates, there may be other explanations for their chemical abundance pattern \citep[e.g.,][]{maiorca2011}. 


\subsubsection{Enhancements due to sodium, technetium, and carbon} \label{sec:other enhancements}
Enhancements in sodium, technetium, and carbon are useful indicators when determining whether a sample is populated by intrinsically enhanced AGB S stars, or polluted extrinsic barium stars. We? performed an identical analysis to the process described in Section \ref{sec:cand} in order to identify sodium, technetium, or carbon enhancement in our sample of 187 s-process candidates. For sodium enrichment, we required significant absorption in the doublet lines at 5889\,\AA\ and 5895\,\AA\. Only 5/187 s-process candidates met this criteria. For technetium enhancement, we searched for significant residual deviations at 4049\,\AA, 4238\,\AA, 4262\,\AA, 4297\,\AA, and 5924\,\AA. These absorption lines are extremely weak, and would require a substantial amount of technetium enhancement before it would be visible in a high S/N LAMOST spectrum. We found that 51 s-process candidates exhibited some level of significant enhancement at the single absorption line 4238\,\AA. Single line enrichment discovered by our data driven approach are very likely attributed to data reduction/calibration artefacts present. These artefacts exist within in our data as a result of fitting a model to the LAMOST spectral data. As such, the 51 stars enhanced in a single Tc line can be ruled out as false positives. For carbon enhancement we searched for significant deviations at the CH and G band near 4300\,\AA, of which 178/187 s-process candidates exhibited significant carbon enhancement (i.e., $[\textrm{C/Fe}] \gtrsim 0.5$). 

\begin{table*}
	\centering
	\caption{Properties of 895 s-process candidates, the table is available online in its entirety. Here we show a portion to demonstrate its style and content.}
	\label{table:table1}
	\begin{adjustbox}{width=1\textwidth}
		\begin{tabular}{@{}|l|l|c|c|c|c|c|c|c|c|c|c|c|c|c|@{}}
			\toprule
			2MASSID             & R.~A.         & Dec.        & $v_{r}$ & S/N & $\rm T_{\rm eff}$ & $\log{g}$ & [Fe/H] & [$\alpha$/Fe] & $\chi_r^2$ & [Ba/Fe] & [Sr/Fe] & \ion{Ba}{II} & \ion{Sr}{II} &  \ion{Ba}{II} \& \ion{Sr}{II} \\
			& (J2000) & (J2000) & (km\,s$^{-1}$) & (pixel$^{-1}$) & (K)  \\ \midrule
			J000019.26+501444.8 & 00:00:19.27 & +50:14:44.9 & -3.9  & 49      & 4973         & 3.27         & 0.21         & 0.08             & 0.66               & 0.25        & 0.83        & \ding{55} & \ding{51}  & \ding{55}   \\
			J000020.55+411348.1 & 00:00:20.56 & +41:13:48.2 & -28.2 & 32      & 4882         & 2.74         & -0.22        & 0.04             & 0.23               & -0.17       & 0.90        & \ding{55}& \ding{51}  & \ding{55}    \\
			J000134.95+490743.2 & 00:01:34.96 & +49:07:43.2 & -42.3 & 72      & 5044         & 3.11         & -0.54        & 0.11             & 0.79               & 1.02        & 0.45        & \ding{55} & \ding{55} & \ding{51}  \\
			J000258.09+410730.0 & 00:02:58.10 & +41:07:30.1 & -34.5 & 41      & 4697         & 2.57         & -0.22        & 0.12             & 0.98               & -0.10       & 0.80        & \ding{55} & \ding{51}  & \ding{55}  \\
			J000403.80+160257.1 & 00:04:03.80 & +16:02:57.2 & -35.4 & 42      & 5200         & 3.40         & -0.41        & 0.09             & 0.33               & 0.92        & 0.52        & \ding{55} & \ding{55} & \ding{51}  \\
			J000439.16+183350.3 & 00:04:39.17 & +18:33:50.4 & -38.7 & 31      & 4601         & 2.50         & 0.44         & 0.03             & 0.34               & -0.11       & 0.88        & \ding{55} & \ding{51}  & \ding{55}  \\
			J000444.87+400402.1 & 00:04:44.88 & +40:04:02.1 & -94.7 & 53      & 4172         & 1.46         & -0.08        & 0.09             & 0.72               & 0.03        & 0.91        & \ding{55} & \ding{51}  & \ding{55}  \\
			J000552.76+261849.3 & 00:05:52.76 & +26:18:49.4 & -22.5 & 73      & 4924         & 3.14         & -0.05        & 0.09             & 0.67               & 0.28        & 0.85        & \ding{55} & \ding{51}  & \ding{55}  \\
			J000737.70+394055.5 & 00:07:37.70 & +39:40:55.6 & -31.5 & 33      & 4700         & 2.80         & -0.04        & 0.07             & 0.28               & -0.13       & 0.81        & \ding{55} & \ding{51}  & \ding{55}  \\ \hline
		\end{tabular}
	\end{adjustbox}
\end{table*}

\subsubsection{Abundances estimated from LAMOST spectra}
We estimated [Ba/Fe] and [Sr/Fe] abundance ratios for all s-process enhanced candidates by synthesising spectra. We assumed that absorption due to metals is captured by \emph{The Cannon} model, and deviations in flux at the 4554\,\AA\ \ion{Ba}{II} line and the 4077\,\AA\ \ion{Sr}{II} transition are solely due to enhancements in Ba and Sr, respectively. We used an extensive grid of model
atmospheres developed by the MARCS program \citep{marcs}, determined stellar parameters using the SME code\citep{sme},? 


In analyses of stellar spectra and colours, and for the analysis of integrated light from galaxies, a homogeneous grid of model
atmospheres of late-type stars and corresponding flux spectra is needed
We construct an extensive grid of spherically-symmetric models (supplemented with plane-parallel ones for the highest surface
gravities), built on up-to-date atomic and molecular data, and make it available for public use.

\citep{marcs,sme,vald,ispec}. We adopted the stellar parameters ($T_{\rm eff}$, $\log{g}$, [Fe/H]) from \citet{ho2017}, and assume a microturbulence of $v_{mic} = 2\,{\rm km\,s}^{-1}$. Uncertainties in [Ba/Fe] and [Sr/Fe] from LAMOST are taken as the fitting error due to noise, added in quadrature with an adopted $0.2\,{\rm dex}$ systematic error floor.

Our analysis shows 186/187 of our s-process candidates exhibit enhancement at the BaII resonance line 4554 with a $\rm[Ba/Fe] \geq +0.3$, and a average abundance error of 0.1. Previous studies \citet{malaney1988} have defined s-process candidates through an abundance criteria of $\rm[Ba/Fe]\geq +0.3$. Using this measure, only one star ($\rm J092155.95+265253.4$) from our sample exhibits a lower [Ba/Fe] of 0.12. This however does not rule out s-process enhancement for this candidate. As previously discussed, a negative residual at the \ion{Ba}{II} resonance line 4554\,\AA\ indicates s-process enhancement and as discussed in section \ref{sec:other enhancements}, carbon enrichment is observed frequently with s-process enhancement. $\rm J092155.95+265253.4$ shows both a negative residual at 4554\,\AA\ and carbon enrichment, we therefore conclude that this single deviant case can be classified as a s-process candidate. This additionally applies to candidates that fall under $\rm[Ba/Fe]\geq +0.3$ due to the average abundance error. 


\subsection{Dynamics}
We integrated the galactic orbits using recent astrometry from Gaia \citep{gaia2016,gaia2018b, cropper2018, katz2018, lindegren2018, sartoretti2018} for 871/895 s-process enhanced stars with parallax > 0 and parallax/parallax error > 5 selection filters. We integrated each star backwards for $0.5\,\textrm{Gyr}$ using the \texttt{gala} Python package \citep{price2017} in a Milky Way-like potential \citep{mwpotential2014} that consists of four components: a \citet{hernquist1990} bulge and nucleus, a \citet{miyamoto1975} disk, and a \citet{nfw1997} halo. We computed spatial velocities relative to the local standard of rest, where $U_{LSR}$ is positive towards the Galactic centre, $V_{LSR}$ is positive in the direction of Galactic rotation ($l=90^{\circ}, b=0^{\circ}$), and $W_{LSR}$ is positive towards the north galactic pole ($b=90^{\circ}$). Our analysis revealed 97\% disk star membership and 3\% with velocity consistent to the halo which is shown using a Toomre diagram in Figure \ref{fig:figure4}.

\section{Follow-up observations with Magellan/MIKE} \label{sec:observations}

\subsection{Observations and data reduction}
On 07 January 2018 we acquired a high-resolution spectrum of two s-process candidates (J09162834+0259348 and J08351472-0548480) using the MIKE (Magellan Inamori Kyocera Echelle) \citep{bernstein2003} spectrograph on the Magellan Clay telescope \citep{schectman2003} at Las Campanas Observatory, Chile. These two candidates were observed as part of another program not a detailed follow-up campaign. They were selected based on observability constraints and ranked by their apparent visual magnitude, with both candidates having $V \approx 12$. We observed both stars in good seeing using the 0.7 arcsecond slit and 2x2 spatial on-chip binning, providing a spectral resolution of $\mathcal{R} \approx 28,000$. Exposure times of 100 seconds were sufficient to achieve a S/N ratio exceeding 30 per pixel at 4500\,\AA. We acquired calibration (biases, milky, quartz, and Th-Ar arc lamp) frames in the afternoon. We reduced the data using the \texttt{CarPy} package \citep{kelson2000}. We used spline functions to continuum-normalise all echelle orders, and resampled the normalised spectra onto a uniform-spaced wavelength map. We used a rest-frame normalised template of a FGK-type star to place the observed spectra at rest.


\subsection{Abundance analysis}
We adopted the stellar parameters ($T_{\rm eff}$, $\log_{10}g$, [Fe/H]) provided from the data-driven analysis of \citet{ho2017}. Following the procedure outlined in \citet{casey2014}, we measured the strength of \ion{Ba}{II} (4554\,\AA, 4934\,\AA, and 6496\,\AA) and \ion{Sr}{II} (4077\,\AA\ and 4215\,\AA) absorption lines by spectral synthesis, using s-process isotopic ratios from \citet{sneden08}. The abundance ratios we estimate from high-resolution spectra are in excellent agreement with our estimates from LAMOST spectra, all agreeing within the joint $1.2\sigma$. Specifically, for J09162834+0259348 from high- and low-resolution spectra, respectively, we find $[{\rm Sr/Fe}] = 0.76 \pm 0.10$ and $0.85 \pm 0.21$, and $[{\rm Ba/Fe}] = 0.92 \pm 0.10$ and $0.77 \pm 0.21$. The biggest discrepancy we find between high- and low-resolution spectra is [Sr/Fe] for J09162834+0259348, where we find $[{\rm Sr/Fe}] = 0.62 \pm 0.07$ from our Magellan/MIKE spectra, and $0.90 \pm 0.22$ from LAMOST. Finally, for J09162834+0259348 we find $[{\rm Ba/Fe}] = 0.94 \pm 0.12$ from high-resolution spectra and $[{\rm Ba/Fe}] = 0.80 \pm 0.26$ from low-resolution spectra. Uncertainties on abundances derived from high-resolution spectra are taken as the standard deviation of multiple line measurements. The high-resolution abundances we find help validate our methodology for candidate selection, and for estimating abundances from LAMOST spectra.


%\subsubsection{Abundance Uncertainties}


\section{Discussion}  \label{sec:dis}


\subsection{Extrinsic or intrinsic}
S-process enrichment can be explained by intrinsic or extrinsic phenomena. Intrinsically enhanced s-process candidates must be massive enough to reach the TP-AGB phase, and extrinsic s-process candidates must be in a binary with a previously polluting TP-AGB star companion (AGB S star). Previous literature \citep{van2017} states only 33\% of their sample are post-main-sequence s-process candidates (extrinsic). In comparison, Figure \ref{fig:figure2} shows that the majority of our sample have temperatures and luminosity not covered by the AGB. This highlights that our sample does not contain stars evolved enough to reach the TP-AGB phase and therefore, must be extrinsic s-process candidates. This is in agreement with our sample selection given we were predominantly restricted to RGB stars. However stars with temperatures below 3800K in Figure \ref{fig:figure2} are consistent with being AGB stars, and their intrinsic or extrinsic nature is not certain. 

\begin{figure}
	\includegraphics[width=0.5\textwidth]{hrd_new.pdf}
    \caption{Effective temperature $T_{\rm eff}$ and surface gravity $\log{g}$ for 187 s-process candidates from LAMOST.}
    %Stars with $\log{g} > 2$ (as marked) have not evolved through the thermally pulsing asymptotic giant branch phase, and therefore their abundance signature must be the result of an extrinsic pollution event. The classification of stars with $\log{g} < 2$ would require additional data.}
    \label{fig:figure2}
\end{figure}

\subsection{Sodium enhancement}
Sodium overabundances observed in the atmospheres of giants are primarily synthesised by the NeNa reaction chain in the convective core of main-sequence stars \citep{el1995}. Sodium is transported to the surface via mixing during the first dredge-up and stars with masses more than $1.5\,M_\odot$ can show enhanced sodium abundances \citep{denissenkov1987,smiljanic2012}. \citet{antipova2004} reported  enhanced sodium in 3 of their 16 barium star candidates. They relate this to the dredge-up of nuclear-burning material produced by convection during the red-giant phase and suggest that [Na/Fe] ratios are systematically higher for giants with lower $\log{g}$ values. Similarly, \citet{decastro2016} proposes a possible weak anti-correlation between [Na/Fe] ratio and $\log{g}$, and highlights that this trend is present in previous studies \citep[e.g.,][]{boyarchuk2002,mishenina2006,luck2007,takeda2008}.

%however, these [Na/Fe] ratios do not exhibit variations in metallicity
We find that less than $3$\,\% of our s-process candidates show enhanced sodium all of which have values of $\log{g} \approx 3$. It is possible our 5/187 enhanced sodium candidates are false-positives resulting from interstellar dust absorption. Using the IRSA's all-sky dust map \citep{schlafly2011}, we find that 2/5 of these candidates have E(B-V)$\approx$ 0.35 confirming that the flux residuals around the Na doublet are due to interstellar absorption. However 3/5 candidates exhibit low ($\approx$0.045) E(B-V) values indicating the enhancement is more likely due to stellar absorption. These results suggest that sodium-rich material from a companion TP-AGB star has polluted these 3 s-process candidates. It also highlights sodium enhancement is by no means ubiquitous in s-process candidates. Yet, low resolution spectra can imitate Na line saturation and limits flux enhancement even if an abundance enhancement is present.
%Previous studies \citep[e.g.,][]{decastro2016} suggest lower $\log{g}$ values systematically contribute to higher [Na/Fe] abundance ratios.

The data that led \citet{decastro2016} to conclude that s-process candidates have higher [Na/Fe] ratios is subject to systematic differences (biases) between Milky Way studies of normal FGK-type stars, and those focussed on s-process candidates. This has the potential to contain unresolved differences in the reported [Na/Fe] abundance ratio. These systematic effects would also contribute to the observed correlation between [Na/Fe] and $\log{g}$, since intrinsic s-process candidates (by definition) have low $\log{g}$ values. We do not observe this correlation between [Na/Fe] and $\log{g}$ in our sample. We find that s-process candidates do not have higher [Na/Fe] abundance ratios on average.

\subsection{Technetium enhancement}

Technetium has a half life of approximately 211,000 years and as a result, the presence of Tc enhancement is constrained to observations of AGB S stars \citep{jorissen1993} or recently polluted extrinsic s-process candidates. Our analysis searches for flux residuals around the Tc enhancement lines 4049\,\AA, 4238\,\AA, 4262\,\AA, 4297\,\AA, and 5924\,\AA. We found 51/187 of our s-process candidates to show enhanced Tc at the single line 4238,\AA. The majority (46/51) have $\log{g} > 2$ suggesting they cannot be AGB S stars and may indicate a recent ($<1\,\textrm{Myr}$) mass accretion event. However we caution that Tc enhancement is usually a very weak signature and would require multiple lines of enhancement. Since all 51 matches were at single lines we concluded Tc enhanced candidates are more likely a result of data artefacts. Also, it is universally known that numerous Tc lines are considerably blended by other s-process enhancement lines \citep[e.g.,][]{van1999}. We therefore conclude our analysis finds no enhanced Tc s-process candidates and is in agreement with previous findings \citep[e.g.,][]{little1987,smith1984,smith1983}. The absence of Tc indicates our s-process candidates are not massive enough to produce the element, and suggests that our sample is populated solely by extrinsic s-process candidates.  


\begin{figure}
	\includegraphics[width=0.5\textwidth]{yields_test.pdf}
    \caption{Metallicities ([Fe/H]; x-axis) and heavy-to-light s-process abundance ratios ([Ba/Sr] as measured from LAMOST; y-axis) for 187 barium star candidates. Coloured lines blue, orange, and green, indicate surface [hs/ls] yields from \citet{cristallo2015} for different masses. Coloured lines red, purple, and brown indicate surface [hs/ls] yields from \citet{karakas_lugaro2016} for [Fe/H] at +0.3. 0.0, and -0.3. From \citet{karakas2018} for [Fe/H] at -0.7, and from \citet{fishlock2014} for [Fe/H] at -1.2.}
    \label{fig:figure3}
\end{figure}

\subsection{Carbon bands}
During the AGB phase of a star, carbon abundance is a product of helium fusion via the triple-alpha process. Giants and supergiants become enhanced in carbon through TDU process discussed in Section \ref{sec:intro}, and extrinsic carbon enhancement occurs via mass transfer from a binary AGB star. Our analysis identifies carbon enhanced candidates by searching for flux residuals in the CH and G band regions at lines 4050\,\AA\hspace{0.5mm} and 4550\,\AA. We found 178/187 of our s-process candidates to have carbon enrichment (i.e., $[\textrm{C/Fe}] \gtrsim 0.5$). These 178 stars have in a metallicities ranging of 0 to -1.2, and are classically referred to as CH stars \citep[e.g.][]{luck1991, mcclure1997}, we simply define them as metal poor s-process candidates.

\subsection{Comparison to AGB yields}
In Figure \ref{fig:figure3} we show the heavy-to-light s-process abundance ratio (taken as [Ba/Sr]) for all 187 s-process candidates identified in LAMOST. Although the [Ba/Sr] ratio is quite noisy ($\sim$0.2\,dex) the overall metallicity [Fe/H] is quite precise (0.1\,dex). The final surface s-process index [Ba/Sr] seen in Figure \ref{fig:figure3} represents degree of the s-process pollution and is an indication of the binary AGB companion's mass. The [Ba/Sr] yields for our s-process candidates follow similar trends to previous studies \citep{fishlock2014,cristallo2015,karakas_lugaro2016}. This suggests that the main neutron source for s-process enhancement in our sample is the $C^{13}(\alpha,n)O^{16}$ reaction chain. Theoretical models show that this source is dominant in low-mass AGB stars and it follows, that most of the progenitors for s-process enhancement in our sample are low-mass AGB stars between $1 - 3\,M_{\odot}$. However our results do not indicate when the mass transfer occurred, and our candidates fitting these theoretical yields may not have the correlating binary companion masses. Mass transfer may have occurred early on in the AGB S binary companion star's lifespan. The resulting [Ba/Sr] abundance present in our s-process candidates and those not approximated by theoretical curves may reflect pollution from a lower AGB S star mass. Our s-process candidates that deviate from this relationship may also represent the possibly more massive and exceptionally rarer AGB S stars within our sample.

\subsection{Dynamics}

The majority of our sample show pro-grade orbits that were consistent with membership in the Milky Way disk. Figures \ref{fig:figure4} and \ref{fig:figure5} illustrate the galactic position of each class of s-process enhanced star investigated. For our entire 898 sample, Figure \ref{fig:figure4} shows 97\% reside within the galactic disk  and 3\% with velocity consistent to the halo. Our results show similar trends to the literature, samples from \cite{gomez1997} and \cite{mennessier1997} present populations of 75\% contained within the disk and 25\% within the halo. \cite{pereira2011}’s entire sample resides within the disk. Although, these studies contain limited populations ranging from 4 to 12 stars, and cannot be representative of the overall galactic distribution of s-process enhanced stars. \cite{decastro2016} contains a more extensive sample and has similar trends to our findings with a 90\% thin disk membership. \cite{jorissen1993} attributes the galactic position of s-process enhanced stars as dependent on their intrinsic or extrinsic nature, concluding intrinsic s-process candidates are more concentrated towards the galactic plane. Since our sample is entirely extrinsic we cannot comment on the comparison between the two enhancement sources. However given our extrinsically s-process enhanced sample concentrates towards the galactic plane, we can propose the sample in \cite{jorissen1993} was not sufficiently large enough to represent the overarching galactic distribution of extrinsic S stars. Or this may be the result of our sample being bias towards finding extrinsic s-process candidates since we were limited mainly to RGB stars.

\begin{figure}
	\includegraphics[width=0.5\textwidth]{toomre.pdf}
	\caption{Galactic distribution represented by a Toomre diagram, spatial velocities ([V (km/s)]; x-axis) and ([$\sqrt{U^2+W^2}$ (km/s)]; y-axis) for number of s-process enhanced candidates matched with Gaia DR2. Coloured markers red, blue, and purple highlight barium, strontium, and both barium and strontium enhancement respectively.}
	\label{fig:figure4}
\end{figure}

\begin{figure}
	\includegraphics[width=0.5\textwidth]{contour.pdf}
	\caption{Galactic longitude ([$l^{\circ}$]; x-axis) and galactic latitude ([$b^{\circ}$]; y-axis) shown for: number of s-process enhanced candidates matched with Gaia DR2, the LAMOST sample matched with Gaia DR2, and a density contour between the latter two}
	\label{fig:figure5}
\end{figure}

\section{Conclusions} \label{sec:con}

We conducted the largest ever search for s-process enhanced stars using the LAMOST second data release. From 454,180 giant stars, we identify 895 s-process enhanced stars including; 187 s-process candidates, 49 barium enriched stars, and 659 stars with strontium enrichment. This sample size is the greatest total number of s-process candidates known, and represents the largest sample of s-process enhanced stars to date.  We found our sample contains 97\% disk star membership and 3\% with velocity consistent to the halo.

We find the majority (178/257) of our s-process candidates show carbon enhancement in agreement with the literature and are classically defined CH stars. However contrary to previous works we do not find our s-process candidates to have significantly higher [Na/Fe] than Milky Way field giants. Only 5/257 of our s-process candidates show enhancement in Na, and the flux residuals in three of those are likely due to interstellar dust. We suggest that biases between literature sources and systematic effects in measuring [Na/Fe] from stars with low $\log{g}$ contributed to this effect. 

Despite our noisy estimates of [Ba/Fe] and [Sr/Fe] from LAMOST spectra, comparisons with AGB yields indicate the main neutron source responsible for s-process enhancement is the $C^{13}(\alpha,n)O^{16}$ reaction chain. Theoretical yields suggest the progenitors of our s-process enhanced sample are low-mass AGB stars between $1 - 3\,M_{\odot}$. We encourage follow-up observations with high-resolution spectrographs in order to precisely measure a full suite of neutron-capture abundances and perform a comprehensive comparison of AGB star models.

 

\section*{Acknowledgements}
We thank David W. Hogg (NYU) and Hans-Walter Rix (MPIA) for useful discussions. 
A.~R.~C. is supported through an Australian Research Council Discovery Project under grant DP160100637.
A. Y. Q. H. was supported by the GROWTH project funded by the National Science Foundation under PIRE Grant No 1545949, and a National Science Foundation Graduate Research Fellowship under Grant No. DGE-1144469. 
This research has made use of NASA's Astrophysics Data System.
Guoshoujing Telescope (the Large Sky Area Multi-Object Fiber Spectroscopic Telescope LAMOST) is a National Major Scientific Project built by the Chinese Academy of Sciences. Funding for the project has been provided by the National Development and Reform Commission. LAMOST is operated and managed by the National Astronomical Observatories, Chinese Academy of Sciences. 
This work has made use of data from the European Space Agency (ESA) mission
{\it Gaia} (\url{https://www.cosmos.esa.int/gaia}), processed by the {\it Gaia}
Data Processing and Analysis Consortium (DPAC,
\url{https://www.cosmos.esa.int/web/gaia/dpac/consortium}). Funding for the DPAC
has been provided by national institutions, in particular the institutions
participating in the {\it Gaia} Multilateral Agreement.


 
\bibliographystyle{mnras}
%\bibliography{example} % if your bibtex file is called example.bib

\begin{thebibliography}{99}
\bibitem[Alam et al.(2015)]{alam2015} Alam, S., Albareti, F.~D., Allende Prieto, C., et al.\ 2015, \apjs, 219, 12 
%\bibitem[\protect\citeauthoryear{Allen \& Barbuy}{2006}]{allen2006}
%Allen, D.~M.,\& Barbuy B. 2006, 
%A$\&$A, 454, 917
\bibitem[\protect\citeauthoryear{Antipova et al.}{2004}]{antipova2004}
Antipova, L.~I., Boyarchuk, A.~A., Pakhomov, Yu.~V.,\& Panchuk, V.~E. 2004, 
ARep, 48, 597
\bibitem[\protect\citeauthoryear{Bernstein et al.}{2003}]{bernstein2003}
Bernstein, R., Shectman, S.~A., Gunnels, S.~M., Mochnacki, S.,\& Athey, A.~E. 2003, 
SPIE, 4841, 1694    
\bibitem[\protect\citeauthoryear{Bidelman \& Keenan}{1951}]{Bidelman1951}
Bidelman, W.~P. \& Keenan, P.~C. , 1951, ApJ, 114, 473
\bibitem[\protect\citeauthoryear{Bisterzo et al.}{2014}]{bisterzo2014}
Bisterzo, S., et al. 2014, 
ApJ, 787, 10
A$\&$A, 424, 727
\bibitem[Blanco-Cuaresma et al.(2014)]{ispec} Blanco-Cuaresma, S., Soubiran, C., Heiter, U., \& Jofr{\'e}, P.\ 2014, \aap, 569, A111 
\bibitem[\protect\citeauthoryear{Boffin \& Jorissen}{1988}]{boffin1988}
Boffin H, M.~J.,\& Jorissen, A. 1988, 
A$\&$A, 205, 155
\bibitem[\protect\citeauthoryear{B\"ohm-Vitense}{1980}]{bohm1980}
B\"ohm-Vitense, E. 1980, 
ApJ, 239, L79
\bibitem[\protect\citeauthoryear{B\"ohm-Vitense et al.}{1984}]{bohm1984}
B\"ohm-Vitense, E., Nemec, J.,\& Proffitt, C. 1984, 
ApJ, 278, 726
\bibitem[\protect\citeauthoryear{Bovy}{2015}]{bovy2015}
Bovy, J. 2015, 
ApJ, 216, 29
\bibitem[\protect\citeauthoryear{Boyarchuk et al.}{2002}]{boyarchuk2002}
Boyarchuk, A.~A., Pakhomov, Y.~V., Antipova, L.~I.,\& Boyarchuk, M.~E. 2002, 
ARep, 46, 819
\bibitem[\protect\citeauthoryear{Busso et al.}{1999}]{busso1999}
Busso, M., Gallino, R., \& Wasserburg, P.~C. 1999, 
ARA$\&$A, 37, 239
\bibitem[\protect\citeauthoryear{Busso et al.}{2001}]{busso2001}
Busso, M., Gallino, R., Lambert, D.~L., Travaglio, C.\& Smith, V.~V. 2001, 
ApJ, 557, 802
\bibitem[\protect\citeauthoryear{Casey}{2014}]{casey2014}
Casey, A.~R. 2014, 
PhD Thesis, Australian National University
\bibitem[\protect\citeauthoryear{Cristallo et al.}{2015}]{cristallo2015}
Cristallo, S., Straniero, O., Piersanti, L.,\& Gobrecht, D. 2015, 
ApJ, 219, 40
\bibitem[\protect\citeauthoryear{Cropper et al.}{2018}]{cropper2018}
Cropper, M., Katz, D., Sartoretti, P., Prusti, T., de Bruijne, J.~H.~J., Chassat, F., Charvet, P., Boyadijan, J., Perryman, M., Sarri, G., Gare, P., Erdmann, M., Munari, U., Zwitter, T., Wilkinson, M., Arenou, F., Vallenari, A., G{\'o}mez, A., Panuzzo, P., Seabroke, G., Allende Prieto, C., Benson, K., Marchal, O., Huckle, H., Smith, M., Dolding, C., Jan{\ss}en, K., Viala, Y., Blomme, R., Baker, S., Boudreault, S., Crifo, F., Soubiran, C., Fr{\'e}mat, Y., Jasniewicz, G., Guerrier, A., Guy, L.~P., Turon, C., Jean-Antoine-Piccolo, A., Th{\'e}venin, F., David, M. Gosset, E.,\& Damerdji, Y. 2018
\bibitem[\protect\citeauthoryear{deCastro et al.}{2016}]{decastro2016}
deCastro, D.~B., Pereira, C.~B., Roig, F., Jilinski, E., Drake, N.~A., Chavero, C.,\& Sales Silva, J.~V. 2016, 
MNRAS, 459, 4299
\bibitem[\protect\citeauthoryear{Denissenkov \& Ivanov}{1987}]{denissenkov1987}
Denissenkov P.~A.,\& Ivanov, V.~V. 1987, 
SvAL, 13, 214
%\bibitem[\protect\citeauthoryear{Drake \& Pereira}{2008}]{drake2008}
%Drake N.,~A.,\& Pereira, C.~B. 2008, 
%AJ, 135, 1070
\bibitem[\protect\citeauthoryear{El Eid \& Champagne}{1995}]{el1995}
El Eid, M.~F.,\& Champagne, A.~E. 1995, 
ApJ, 451, 298
\bibitem[\protect\citeauthoryear{Fishlock et al}{2014}]{fishlock2014}
Fishlock, C.~K., Karakas, A.~I., Lugaro, M.,\& Yong, D. 2014, 
ApJ, 797, 44
\bibitem[\protect\citeauthoryear{Gaia Collaboration et al.}{2016}]{gaia2016}
Gaia Collaboration, Prusti, T., de Bruijne, J.~H.~J., Brown, A.~G.~A., Vallenari, A., Babusiaux, C., Bailer-Jones, C.~A.~L., Bastian, U., Biermann, M.,\& Evans, D.~W. et al. 2016, 
A$\&$A, 595, A1
\bibitem[\protect\citeauthoryear{Gaia Collaboration et al.}{2018b}]{gaia2018b}
Gaia Collaboration, Brown, A.~G.~A., Vallenari, A, Prusti, T., de Bruijne, J.~H.~J., Babusiaux, C.,\& Bailer-Jones, C.~A.~L. 2018, 
\bibitem[Gustafsson et al.(2008)]{marcs} Gustafsson, B., Edvardsson, B., Eriksson, K., et al.\ 2008, \aap, 486, 951 
\bibitem[\protect\citeauthoryear{Gomez et al.}{1997}]{gomez1997}
Gomez, A.~E., Luri, X., Grenier, S., et al. 1997, 
A$\&$A, 319, 881
\bibitem[\protect\citeauthoryear{Han et al.}{1995}]{han1995}
Han, Z., Eggleton, P.~P., Podsiadlowski, P.,\& Tout, C.~A. 1995, 
MNRAS, 277, 1443
\bibitem[\protect\citeauthoryear{Hernquist}{1990}]{hernquist1990}
Hernquist, L. 1990, 
ApJ, 356, 359
\bibitem[\protect\citeauthoryear{Herwig}{2005}]{herwig2005}
Herwig, F. 2005, 
ARA$\&$A, 43, 435
\bibitem[\protect\citeauthoryear{Ho et al.}{2017}]{ho2017}
Ho, A.~Y.~Q., Ness, M.~K., Hogg, D.~W., Rix, H.-W. Liu, C., Yang, F., Zhang, Y., Hou, Y.,\& Wang, Y. 2017, 
ApJ, 836, 5
\bibitem[\protect\citeauthoryear{Jorissen \& Boffin}{1992}]{jorissen1992}
Jorissen, A.,\& Boffin H, M.~J., 1992, 
Evidences for interaction among wide binary systems: To Ba or not to Ba? In: Duquennoy, A., Mayor, M.,(eds.) Binaries as tracers of stellar formation. Cambridge Univ. Press., p.185
\bibitem[\protect\citeauthoryear{Jorissen et al.}{1993}]{jorissen1993}
Jorissen, A., Frayer, D.~T., Johnson, H.~W., Mayor, M.,\& Smith, V.~V. 1993, 
A$\&$A, 271, 463
\bibitem[\protect\citeauthoryear{Jorissen et al.}{2005}]{jorissen2005}
Jorissen, A., Za$\check{c}$, L., Udry, S., Lindgren, H.,\& Musaev, F.~A. 2005, 
A$\&$A, 441, 1135
%\bibitem[\protect\citeauthoryear{Junqueira \& Pereira}{2001}]{junqueira2001}
%Junqueira S.,\& Pereira, C.~B. 2001, 
%AJ, 122, 360
\bibitem[\protect\citeauthoryear{Karakas et al.}{2002}]{karakas2002}
Karakas, A.~I.,\& Lattanzio C.~J.,\& Pols, O.~R. 2002, 
PASA, 515, 19
\bibitem[\protect\citeauthoryear{Karakas \& Lattanzio}{2014}]{karakas2014}
Karakas, A.~I.,\& Lattanzio C.~J. 2014, 
PASA, 31, 30
\bibitem[\protect\citeauthoryear{Karakas \& Lugaro}{2016}]{karakas_lugaro2016}
Karakas, A.~I.,\& Lugaro M. 2016, 
ApJS, 825, 26
\bibitem[\protect\citeauthoryear{Karakas et al.}{2016}]{karakas2016cp}
Karakas, A.~I. 2016, 
S.A.Lt, 229, 87
\bibitem[\protect\citeauthoryear{Karakas}{2018}]{karakas2018}
Karakas, A.~I. 2018, 
MNRAS, submitted
\bibitem[\protect\citeauthoryear{Katz et al.}{2018}]{katz2018}
Katz, D., Sartoretti, P., Cropper, M., Panuzzo, P., Seabroke, G.~M., Viala, Y., Benson, K., Blomme, R., Jasniewicz, G., Jean-Antoine, A., Huckle, H., Smith, M., Baker, S., Crifo, F., Damerdji, Y., David, M., Dolding, C., Fr{\'e}mat, Y., Gosset, E., Guerrier, A., Guy, L.~P., Haigron, R., Jan{\ss}en, K., Marchal, O., Plum, G., Soubiran, C., Th{\'e}venin, F. ,Ajaj, M., Allende Prieto, C., Babusiaux, C., Boudreault, S., Chemin, L., Delle Luche, C., Fabre, C., Gueguen, A., Hambly, N.~C., Lasne, Y., Meynadier, F., Pailler, F., Panem, C., Royer, F., Tauran, G., Zurbach, C., Zwitter, T., Arenou, F., Bossini, D., Gomez, A., Lemaitre, V., Leclerc, N., Morel, T., Munari, U., Turon, C., Vallenari, A.,\& {\v Z}erjal, M. 2018
\bibitem[\protect\citeauthoryear{Kelson et al.}{2000}]{kelson2000}
Kelson, D.~D., Illingworth, G.~D., van Dokkum, P.~G.,\& Franx, M. 2000, ApJ, 531, 159
%\bibitem[\protect\citeauthoryear{Kelson}{2003}]{kelson2003}
%Kelson, D.~D. 2003, 
%PASP, 115, 688
%\bibitem[\protect\citeauthoryear{Kobayashi et al.}{2011}]{kobayashi2011}
%Kobayashi, C., Karakas, A., \& Umeda, H. 2011, 
%MNRAS, 414, 3231
\bibitem[Kupka et al.(1999)]{vald} Kupka, F., Piskunov, N., Ryabchikova, T.~A., Stempels, H.~C., \& Weiss, W.~W.\ 1999, \aaps, 138, 119 
\bibitem[\protect\citeauthoryear{Lindegren et al.}{2018}]{lindegren2018}
Lindegren, L., Hernandez, J., Bombrun, A., Klioner, S., Bastian, U., Ramos-Lerate, M., de Torres, A., Steidelmuller, H, Stephenson, C., Hobbs, D., Lammers, U., Biermann, M. ,Geyer, R., Hilger, T., Michalik, D., Stampa, U., McMillan, P.~J., Castaneda, J., Clotet, M., Comoretto, G., Davidson, M., Fabricius, C., Gracia, G., Hambly, N.~C., Hutton, A., Mora, A., Portell, J., van Leeuwen, F., Abbas, U., Abreu, A., Altmann, M., Andrei, A., Anglada, E., Balaguer-Nunez, L., Barache, C., Becciani, U., Bertone, S., Bianchi, L., Bouquillon, S., Bourda, G., Brusemeister, T., Bucciarelli, B., Busonero, D., Buzzi, R., Cancelliere, R., Carlucci, T., Charlot, P., Cheek, N., Crosta, M., Crowley, C., de Bruijne, J., de Felice, F., Drimmel, R. Esquej, P., Fienga, A., Fraile, E., Gai, M., Garralda, N., Gonzalez-Vidal, J.~J., Guerra, R., Hauser, M., Hofmann, W., Holl, B., Jordan, S., Lattanzi, M.~G., Lenhardt, H., Liao, S., Licata, E., Lister, T., Loffler, W., Marchant, J., Martin-Fleitas, J.-M., Messineo, R., Mignard, F., Morbidelli, R., Poggio, E., Riva, A., Rowell, N., Salguero, E., Sarasso, M., Sciacca, E., Siddiqui, H., Smart, R.~L., Spagna, A., Steele, I., Taris, F., Torra, J., van Elteren, A., van Reeven, W.,\& Vecchiato, A. 2018
\bibitem[\protect\citeauthoryear{Little-Marenin \& Little}{1987}]{little1987}
Little-Marenin, I.~R.,\& Little, S.~J. 1987, 
AJ, 93, 1539
\bibitem[\protect\citeauthoryear{Luck \& Bond}{1991}]{luck1991}
Luck R.~E.,\& Bond, H.~E. 1991, 
ApJS, 77, 515
\bibitem[\protect\citeauthoryear{Luck \& Heiter}{2007}]{luck2007}
Luck R.~E.,\& Heiter, U. 2007, 
AJ, 133, 2464
\bibitem[\protect\citeauthoryear{Luo et al}{2015}]{luo2015}
Luo, A.~L., Bai, Z.~R., et al. 2015, 
RAA, in press
\bibitem[Maiorca et al.(2011)]{maiorca2011} Maiorca, E., Randich, S., Busso, M., Magrini, L., \& Palmerini, S.\ 2011, \apj, 736, 120 
\bibitem[\protect\citeauthoryear{Malaney \& Lambert}{1988}]{malaney1988}
Malaney, R.~A.,\& Lambert, D.~L. 1988, 
MNRAS, 235, 695
\bibitem[\protect\citeauthoryear{McClure}{1983}]{mcclure1983}
McClure, R.~D. 1983, 
ApJ, 268, 264
\bibitem[\protect\citeauthoryear{McClure}{1997}]{mcclure1997}
McClure, R.~D. 1997, 
PASP, 109, 536
\bibitem[\protect\citeauthoryear{Mennessier et al.}{1997}]{mennessier1997}
Mennessier, M.~O., Luri, X., Figueras, F., et al. 1997, 
A$\&$A, 326, 722
\bibitem[\protect\citeauthoryear{Mishenina et al.}{2006}]{mishenina2006}
Mishenina, T.~V., Bienaym\' e, O., Gorbaneva, T.~I.,\& Charbonnel, C. 2006, 
A$\&$A, 456, 1109
\bibitem[\protect\citeauthoryear{Miyamoto-Nagai}{1975}]{miyamoto1975}
Miyamoto, M,\& Nagai, R. 1975, 
PASJ, 27, 533
\bibitem[\protect\citeauthoryear{Navarro, Frenk \& White}{1997}]{nfw1997}
Navarro, J.~F., Frenk, C.~S.,\& White, S.~D.~M. 1997, 
ApJ, 490, 493
\bibitem[\protect\citeauthoryear{Ness et al.}{2016}]{ness2016}
Ness M., Hogg D.~W., Rix H.~W., Martig M., Pinsonneault M.~H., Ho A. 2016, 
ApJ, 823, 114
%\bibitem[\protect\citeauthoryear{Pereira \& Drake}{2009}]{pereira2009}
%Pereira, C.~B.,\& Drake N.,~A. 2009, 
%A$\&$A, 496, 791
\bibitem[\protect\citeauthoryear{Pereira et al.}{2011}]{pereira2011}
Pereira, C.~B., Sales Silva, J.,~A., Chavero, C., Roig, F.,\& Jilinski E. 2011, 
A$\&$A, 533, A51
\bibitem[\protect\citeauthoryear{Pourbaix et al.}{2004}]{pourbaix2004}
Pourbaix, D., Tokovinin, A.~A., Batten, A.~H., Fekel, F.~C., Hartkopf, W.~I. et al. 2001, 
%\bibitem[\protect\citeauthoryear{Prantzos}{2012}]{prantzos2012}
%Prantzos, N. 2012, 
%A$\&$A, 542, A67
\bibitem[\protect\citeauthoryear{Price-Whelan}{2014}]{mwpotential2014}
Price-Whelan, A.~M., Hogg, D.~W., Johnston, K.~V.,\& Hendel, D. 2014, 
Apj, 4, 794
\bibitem[\protect\citeauthoryear{Price-Whelan}{2017}]{price2017}
Price-Whelan, A.~M. 2017, 
The Journal of Open Source Software, 2, 388
%\bibitem[\protect\citeauthoryear{Romano et al.}{2010}]{romano2010}
%Romano, D., et al. 2010, 
%A$\&$A, 522, A32
\bibitem[\protect\citeauthoryear{Sartoretti et al.}{2018}]{sartoretti2018}
Sartoretti, P., Katz, D., Cropper, M., Panuzzo, P., Seabroke, G.~M. ,Viala, Y., Benson, K., Blomme, R., Jasniewicz, G., Jean-Antoine, A., Huckle, H., Smith, M., Baker, S., Crifo, F., Damerdji, Y., David, M., Dolding, C., Fremat, Y., Gosset, E., Guerrier, A., Guy, L.~P., Haigron, R., Janssen, K., Marchal, O., Plum, G., Soubiran, C., Thevenin, F., Ajaj, M., Allende Prieto, C., Babusiaux, C., Boudreault, S., Chemin, L., Delle Luche, C., Fabre, C., Gueguen, A., Hambly, N.~C., Lasne, Y., Meynadier, F., Pailler, F., Panem, C., Riclet, F., Royer, F., Tauran, G., Zurbach, C., Zwitter, T., Arenou, F., Gomez, A., Lemaitre, V., Leclerc, N., Morel, T., Munari, U., Turon, C.,\& Zerjal, M. 2018
\bibitem[\protect\citeauthoryear{Schectman \& Johns}{2003}]{schectman2003}
Shectman, S.~A.,\& Johns, M. 2003, 
SPIE, 4837, 910
\bibitem[\protect\citeauthoryear{Schlafly \& Finkbeiner}{2011}]{schlafly2011}
Schlafly, E.~F.,\& Finkbeiner, D.~P. 2011, 
Apj, 737, 103
\bibitem[\protect\citeauthoryear{Smiljanic}{2012}]{smiljanic2012}
Smiljanic, R. 2012, 
MNRAS, 422, 1562
\bibitem[\protect\citeauthoryear{Smith \& Wallerstein}{1983}]{smith1983}
Smith, V.~V.,\& Wallerstein, G. 1983, 
ApJ, 273, 742
\bibitem[\protect\citeauthoryear{Smith}{1984}]{smith1984}
Smith, V.~V. 1984, 
A$\&$A, 132, 326
\bibitem[Sneden et al.(2008)]{sneden08} Sneden, C., Cowan, J.~J., \& Gallino, R.\ 2008, \araa, 46, 241 
\bibitem[\protect\citeauthoryear{Takeda et al.}{2008}]{takeda2008}
Takeda, Y., Sato, B.~V.,\& Murata, D. 2008, 
AJ, 60, 781
\bibitem[\protect\citeauthoryear{Travaglio et al.}{2001}]{travaglio2001}
Travaglio, C., et al. 2001, 
ApJ, 549, 346
\bibitem[\protect\citeauthoryear{Webbink}{1986}]{webbink1986}
Webbink, R.~F. 1986, 
In: Leung, K.~C., Zhai, D.~S.(eds.) Critical Observations versus Physical Models for Close Binary Systems. Gordon and Breach, New York, p.403
\bibitem[\protect\citeauthoryear{Whitelock et al.}{2013}]{whitelock2013}
Whitelock, P.~A., et al. 2013, 
MNRAS, 428, 2216
\bibitem[Valenti \& Piskunov(1996)]{sme} Valenti, J.~A., \& Piskunov, N.\ 1996, \aaps, 118, 595 
\bibitem[\protect\citeauthoryear{Van Eck \& Jorissen}{1999}]{van1999}
Van Eck, S. \& Jorissen, A. 1999, 
A$\&$A, 345, 127
\bibitem[\protect\citeauthoryear{Van der Swaelmen et al.}{2017}]{van2017}
Van der Swaelmen, M., Boffin, H.~M.~J., Jorissen, A.,\& Van Eck, S. 2017, 
A$\&$A, 597, A68
\end{thebibliography}
% Don't change these lines
\bsp	% typesetting comment
\label{lastpage}
\end{document}