
\documentclass[a4paper,fleqn,usenatbib]{mnras}

\usepackage{newtxtext,newtxmath}
% Depending on your LaTeX fonts installation, you might get better results with one of these:
%\usepackage{mathptmx}
%\usepackage{txfonts}

\usepackage[T1]{fontenc}
\usepackage{ae,aecompl}
\usepackage{booktabs}


\usepackage{graphicx}	% Including figure files
\usepackage{amsmath}	% Advanced maths commands
\usepackage{amssymb}	% Extra maths symbols
\usepackage{color}

%%%%%%%%%%%%%%%%%%%%%%%%%%%%%%%%%%%%%%%%%%%%%%%%%%

%%%%% AUTHORS - PLACE YOUR OWN COMMANDS HERE %%%%%
\usepackage{float}
\newcommand{\todo}[1]{\textcolor{red}{#1}}
\newcommand{\countertodo}[1]{\textcolor{green}{#1}}

\makeatletter
\newcommand{\rmnum}[1]{\romannumeral #1}
\newcommand{\Rmnum}[1]{\expandafter\@slowromancap\romannumeral #1@}
\makeatother

% Please keep new commands to a minimum, and use \newcommand not \def to avoid
% overwriting existing commands. Example:
%\newcommand{\pcm}{\,cm$^{-2}$}	% per cm-squared

%%%%%%%%%%%%%%%%%%%%%%%%%%%%%%%%%%%%%%%%%%%%%%%%%%

\title[S-process stars in LAMOST]{Discovery of s-process stars in the LAMOST survey}

% The list of authors, and the short list which is used in the headers.
% If you need two or more lines of authors, add an extra line using \newauthor
\author[B.J. Norfolk et al.]{B.J. Norfolk,$^{1}$\thanks{E-mail: bjee7@student.monash.edu (MU)}
A.R. Casey,$^{1,2}$
M.T. Miles,$^{1}$
A.J. Kemp,$^{1}$ 
A.I. Karakas,$^{1}$ \newauthor
J.C. Lattanzio,$^{1}$
and K.C. Schlaufman 
\\
$^{1}$School of Physics \& Astronomy, Monash University, Clayton 3800, Victoria, Australia\\
$^{2}$Faculty of Information Technology, Monash University, Clayton 3800, Victoria, Australia\\
}
\date{Accepted XXX. Received YYY; in original form ZZZ}

\pubyear{2018}

\begin{document}
\label{firstpage}
\pagerange{\pageref{firstpage}--\pageref{lastpage}}
\maketitle

\begin{abstract}
Stars with peculiar enhancements of carbon and heavy elements ($Z > 30$) are historically referred to as Barium stars. While this chemical abundance pattern can be explained by thermally pulsing asymptotic giant branch (AGB) stars, many Barium stars have not evolved through the red giant phase and therefore cannot be AGB stars. These stars are generally thought to have accreted mass from a more evolved AGB companion. For this reason the frequency and properties of less-evolved Barium stars is uniquely informative on the binary fraction, AGB yields, mass transfer, and stellar evolution. Here we present the discovery of 967 s-process candidates including; 257 Barium stars, 49 stars exhibiting exclusively barium enhancement, and 661 exhibiting exclusively strontium enhancement from 454,180 giant stars observed by LAMOST. This sample constitutes the largest sample of s-process enhanced stars discovered to date, and the most comprehensive search for Barium stars in the Milky Way. \todo{Nearly all} of our Barium stars are disk stars, and we find an \todo{steady/increasing/decreasing} fraction of Barium stars with stellar metallicity, as \todo{expected} from AGB yields and stellar evolution theory. 51 Barium stars exhibited tecnetium enhancement in a single Tc emission line suggesting the exsistence of AGB stars in our sample. Most (178/257) of our sample show strong carbon enhancement consistent with the literature. Despite previous claims indicating that Barium stars are preferentially enhanced in sodium due to the NeNa nucleosynthesis cycle occurring in \todo{massive} AGB stars, only 5 ($<2$\,\%) of our Barium stars show any evidence of enhancement in sodium. \todo{This implies ....}
\end{abstract}

\begin{keywords}
keyword1 -- keyword2 -- keyword3
\end{keywords}

\section{Introduction} \label{sec:intro}

Barium stars as first recognized by \citet{Bidelman1951}, are chemically unique objects that exhibit envelopes with an overabundance of carbon and heavy elements (Z > 30) in comparison to the Sun. Barium stars exist as either intrinsic or extrinsic objects; intrinsic Barium stars are in the TP-AGB (thermally pulsing-asymptotic giant branch) phase, where the overabundance of both carbon and heavy elements are produced through the s-process and exist on the surface as the result of a thermal pulsing cycle. According to the mass-transfer hypothesis, extrinsic Barium stars are a consequence of stellar wind accretion \citep{boffin1988,jorissen1992} or Roche-lobe overflow \citep{webbink1986}, and are within a binary system containing a previous TP-AGB companion star (intrinsic Barium star) in its final phase as a white dwarf \citep{bohm1980,bohm1984}. In fact, \citet{mcclure1983} determines 85\% of all Barium stars are in binary systems and claims that those that appear singular are actually pole-on, or highly eccentric binaries with significant radial velocity variations only occuring in a small phase range as detailed by \citet{pourbaix2004}. For these reasons, the properties and occurrence rate of Barium stars are informative of the theoretical birth rate of AGB stars, the binary star fraction as a function of metallicity, the mass ratio of binary stars, as well as AGB yields across different masses and metallicities. 

The slow neutron capture process (s-process), occuring in the interior of AGB stars, synthesizes roughly half of all elements heavier than iron \citep[e.g.][]{busso1999,travaglio2001,herwig2005,romano2010,kobayashi2011,prantzos2012,bisterzo2014,karakas12016}. Late in the AGB phase, during the thermally pulsing-AGB (TP-AGB) phase, thermal instabilities occur in the He shell every $10^5$ years or so, depending on the H-exhausted core mass. These energy bursts drive a convective zone that sweeps the entire region lying between the core mass and He-shell; mixing the products of nucleosynthesis within these regions. The energy from the thermal pulses force the star to expand, pushing the H-shell out to cooler regions and allowing the convective envelope to move inwards to regions previously mixed by the thermal pulse driven convective zones. This expansion and resultsing inward movement is classified as the third dredge up (TDU), and is theorised to occur after each thermal pulse. During the TP-AGB phase, enrichent on the surface in $^{12}$C and heavy elements produced by the s-process is a result of the repeated downwash extensions of TDU \citep[e.g.][]{busso2001}. The star contracts PDU and reignites the H-shell producing the majority of the surface luminosity for the next interpulse period. The interpulse, thermal pulse, and dredge up cycle may occur numerous times and is dependent on the intitial mass, composition, and mass-loss rate of the star.

Barium giants form within a metallicity-dependent initial mass range (approximately 0.8 - 8$M_{\odot}$), with the minimum mass for core helium and carbon burning decreasing as metallicity decreases. The age of these stars vary considerably, with observable stars in metal-poor globular clusters hosting populations of $\approx$12Gyr. In contrast, metal rich, young observable stars may only reach $\approx$100Myr, this includes stars that are at the core carbon burning limit or very close to it \citep[e.g.][]{whitelock2013}.

Many Barium stars have been discovered in the disk and halo of the Milky Way \citep{gomez1997,mennessier1997}. These studies show that the stellar populations of Barium stars can be separated into two groups when considering their absolute luminosity, kinematic, and spatial properties. Barium stars populating the halo have already been investigated \citep[e.g.][]{junqueira2001,drake2008,pereira2009,allen2006} and their relationship with metal-poor, yellow, symbiotic halo stars has been established by \citet{jorissen2005} and \citet{pereira2009}. Additionally, \citet{pereira2011} concludes that metal-rich Barium stars share similar kinematics to other metal-rich and super metal-rich stars already analysed, suggesting that they do not belong to the bulge population. Furthermore, it is necessary to establish known abundances of s-process elements in a greater sample of Barium stars in order to compare their birth rate to theoretical expectations \citep{han1995}, explore the binary fraction at different metallicities, and compare to AGB star yields at different masses and metallicities.

In this paper we analyse 454,180 giant stars from LAMOST data release (DR) 2 to identify and examine 967 s-process candidates including Barium stars, and exclusvely enhanced barium and strontium stars, respectively. Specifcally, the Barium star giants studied here display especially strong s-process enhancement for \ion{Ba}{II} and \ion{Sr}{II} at lines 4554\AA \hspace{0.2mm} and 4077\AA, respectively. Our exclusive barium and strontium candidates show s-process enhancement in 4554\AA \hspace{0.2mm} and 4934\AA for barium enhancement, and, 4554\AA \hspace{0.2mm} and 4215\AA for strontium enhancement. In Section \ref{sec:methods} we describe the observations and candidate selection, as well as the analysis follow up our follow-up observing campaign with Magellan. In Section \ref{sec:dis} we discuss the properties of our Barium stars in context of existing literature. We provide concluding remarks in Section \ref{sec:con}

\section{Methods} \label{sec:methods}
\subsection{LAMOST analysis}
\subsubsection{Data-driven analysis}
The LAMOST (Large sky Area Multi-Object Fibre Spectrographic) survey released low-resolution ($\mathcal{R} \approx 1800$) optical spectra (3700\,\AA\ to 9000\,\AA) for over two million stars in their second data release \citep{luo2015}. Stellar parameters ($T_{\rm eff}$, $\log{g}$, [M/H], [$\alpha$/M]) were derived using a data-driven model constructed using high fidelity APOGEE labels of 9,952 giant stars in common between LAMOST and the APOGEE survey. It is important to note, this sample is exclusive to giant stars presenting the possibility that this research missed dwarf s-process enhanced stars. Cross-validation experience shows that the typical uncertainties are 70K in effective temperature $T_{eff}$, 0.1 dex in surface gravity log g, 0.1 dex in metallicity [Fe/H], and 0.04 dex in the alpha abundance relative to iron [$\alpha$/M]. These uncertainties are approximately comparable to the conservative APOGEE DR12 uncertainties.

\subsubsection{Candidate selection} \label{sec:cand}
The Barium stars analysed in this research were obtained by filtering for significant flux residuals, taken as, a disparity between the normalised LAMOST flux and the data driven model derived by \citet{ho2017}. A negative residual as shown in the randomly selected Figure \ref{fig:figure1}, illustrates an enhancement in \ion{Ba}{II} and \ion{Sr}{II} at lines 4554\AA \hspace{0.2mm}, 4077\AA. These \ion{Ba}{II} and \ion{Sr}{II} resonance lines are very strong, allowing us to identify stars enhanced in neutron-capture elements given a well-fit model for the data. We used five filters that each spectrum, for both enhancement lines, had to meet in order to be considered a s-process candidate; these included:

\renewcommand\labelenumi{(\roman{enumi})}
\renewcommand\theenumi\labelenumi

\begin{enumerate} 
\item Profile amplitude for both enhancement lines must exceed A < -0.05.
\item Both amplittudes must be measured within 3$\sigma$ ($|A|/\sigma _A$ < 3).
\item The wavelength at each absorption line must be within $2$\AA \hspace{0.2mm}($\lambda$ < 2).
\item Reduced $\chi^2$ from The Cannon must be < 3.
\item And, the LAMOST spectra must have a signal-to-noise ratio of S/N > 30 per pixel.
\end{enumerate}
In addition, a visual inspection was undertaken to exclude any results containing, false positives, candidates exhibiting data reduction issues, apparent absorption finer then the spectral resolution, and overly noisy normalised LAMOST spectra. This combined method generated 967 s-process candidates including; 257 Barium stars, 49 stars exhibiting exclusively barium enhancement, and 661 exhibiting exclusively strontium enhancement, as listed in Table \ref{table:table1}. 

\begin{figure}
	% To include a figure from a file named example.*
	% Allowable file formats are eps or ps if compiling using latex
	% or pdf, png, jpg if compiling using pdflatex
	\includegraphics[width=\columnwidth]{posterchild.png}
    \caption{Random Barium Star Spectrum}
    \label{fig:figure1}
\end{figure}

\begin{table*}
\centering
\caption{Available online in its entirety. Here we show a portion to demonstrate its style and content}
\label{table:table1}
\begin{tabular}{@{}|l|l|l|l|l|l|l|l|l|l|@{}}
\toprule
2MASSID             & RA (hms         & DEC (hms)        & Vr (km/s) & S/N & Teff (K) & Log g & {[}Fe/H{]} & {[}$\alpha$/H{]} & $\chi^2$  \\ \midrule
J000134.95+490743.2 & 00:01:34.96 & +49:07:43.2 & -42.27    & 72  & 5044 & 3.11  & -0.54      & 0.11      & 0.79 \\ \midrule
J000403.80+160257.1 & 00:04:03.80 & +16:02:57.2 & -35.38    & 42  & 5200 & 3.40  & -0.41      & 0.09      & 0.33 \\ \midrule
J000908.48+421615.4 & 00:09:08.49 & +42:16:15.5 & -34.78    & 36  & 4825 & 3.02  & -0.46      & 0.16      & 0.25 \\ \midrule
J001754.17+394140.0 & 00:17:54.17 & +39:41:40.0 & -42.87    & 39  & 5008 & 3.16  & -0.64      & 0.12      & 0.23 \\ \midrule
J001841.57+040533.7 & 00:18:41.57 & +04:05:33.8 & -20.69    & 141 & 4840 & 3.68  & -0.52      & 0.06      & 0.93 \\ \midrule
J002107.60+361533.0 & 00:21:07.60 & +36:15:33.1 & -53.06    & 44  & 4854 & 2.87  & 0.01       & 0.06      & 0.54 \\ \midrule
J002537.59+411123.7 & 00:25:37.59 & +41:11:23.8 & -12.59    & 36  & 5075 & 3.30  & -0.32      & 0.10      & 0.33 \\ \midrule
J003755.55+571706.8 & 00:37:55.55 & +57:17:06.9 & -5.70     & 49  & 4046 & 1.14  & -0.09      & -0.01     & 0.83 \\ \midrule
J003813.23+423025.9 & 00:38:13.23 & +42:30:25.9 & -40.17    & 36  & 4475 & 2.12  & 0.14       & 0.00      & 0.51 \\ \midrule
J003959.14+394614.6 & 00:39:59.14 & +39:46:14.6 & 5.10      & 35  & 4955 & 3.23  & -0.40      & 0.11      & 0.28 \\ \midrule
J004119.68+324627.8 & 00:41:19.68 & +32:46:27.9 & 4.50      & 122 & 4861 & 2.59  & -0.31      & 0.08      & 1.67 \\ \midrule
J004156.56+390945.6 & 00:41:56.57 & +39:09:45.6 & -254.52   & 44  & 4771 & 2.34  & -0.74      & 0.16      & 0.57 \\ \midrule
J004743.01+433358.1 & 00:47:43.02 & +43:33:58.2 & -62.66    & 65  & 4834 & 2.64  & -0.60      & 0.17      & 1.06 \\ \midrule
J005101.98+314045.6 & 00:51:01.98 & +31:40:45.7 & -23.68    & 98  & 5202 & 3.32  & -0.53      & 0.10      & 0.67 \\ \midrule
J005254.69+203554.7 & 00:52:54.69 & +20:35:54.7 & 35.68     & 113 & 5026 & 3.17  & -0.82      & 0.20      & 1.01 \\ \midrule
J010836.26+444435.9 & 01:08:36.26 & +44:44:35.9 & 12.59     & 49  & 5189 & 3.19  & -0.56      & 0.10      & 0.98 \\ \midrule
J011919.20+195227.0 & 01:19:19.21 & +19:52:27.0 & -35.38    & 31  & 4897 & 2.77  & -0.54      & 0.11      & 0.29 \\ \midrule
J011930.90-015027.3 & 01:19:30.91 & -01:50:27.4 & -143.30   & 44  & 5040 & 3.22  & -0.91      & 0.20      & 0.48 \\ \midrule
J020316.40+462211.1 & 02:03:16.41 & +46:22:11.1 & 11.09     & 37  & 4897 & 2.89  & -0.39      & 0.11      & 0.51 \\ \bottomrule
\end{tabular}
\end{table*}


\subsubsection{Further enhancements}
We undertook an identical analysis to the process described in Section \ref{sec:cand} for sodium, technetium, and carbon enhancement. We filtered for negative residuals in the enhanced sodium doublet lines 5889.950\AA\hspace{0.2mm} and 5895.924\AA. Our analysis produced 5 matches with our Barium star candidates. 51 canditdates were discovered to be enhanced in a single technetium line. And, for the CH and G band region at approximately 4300\AA\hspace{0.2mm} we matched 178 enhanced stars out of our total 257 stars. 

\subsection{Follow-up observations with Magellan/MIKE}

\subsubsection{Observations}
On the night of \todo{X} January 2018 we performed follow-up observations of \todo{X} Barium star candidates using the MIKE spectrograph on the Magellan Clay telescope at Las Campanas Observatory, Chile. Candidates for follow-up observations were chosen based on observability constraints and ranked by their apparent visual magnitude. We observed \todo{X} stars in good seeing using the 0.7 arcsecond slit and 2x2 spatial on-chip binning, which provides a spectral resolution of ($\mathcal{R} \approx \todo{22,000}$). Exposure times were set to achieve a S/N ratio of 30 per pixel at 4500\,\AA. We acquired calibration (biases, milky, quartz, and Th-Ar arc lamp) frames in the afternoon.

We reduced the data using the \texttt{CarPy} Python package \citep{kelson2000,kelson2003}. We used spline functions to continuum-normalise individual echelle orders, and resampled the normalised spectra onto a uniform-sized wavelength map. We used a rest-frame normalised template of a FGK-type star to place the observed spectra in the rest frame.


\subsubsection{Abundance analysis}
We adopted the stellar parameters ($T_{\rm eff}$, $\log_{10}g$, [Fe/H]) provided from the data-driven analysis of \citep{ho2017}. Following the procedure outlined in \citep{casey2014}, we measured the strength of atomic absorption lines by fitting Gaussian profiles to the continuum-normalised spectra. Most elemental abundances were derived directly from these line measurements, except for elements that show significant hyperfine splitting \todo{(e.g., )} or blending \todo{(e.g., X)}. In these situations, we adopted a spectral synthesis approach. \todo{citations for line lists}


\todo{Some sentences for our fearless leader to add regarding abundance analysis and verification of stellar parameters from LAMOST}

%\subsubsection{Abundance Uncertainties}

\subsection{Dynamics}
The dynamical analysis undertaken for our Barium star candidates was accomplished by crossing matching the sample with Gaia \citep{gaia2016a,gaia2016b} to produce 21 matches with proper motions, parallax, and galactic positions. We then, using the gala Python package, integrated our data to calculate the orbits of our 21 candidates with a Milky Way Potential model. The Milky Way potential implemented was constructed using a four component potential model built from a \citet{hernquist1990} bulge and nucleus, a \citet{miyamoto1975} disk, and the \citet{nfw1997} halo. In this potential, the disk and bulge parameters were adjusted to be consistent with previous work \citep{bovy2015}, and to vary the scale mass and radius, of the nucleus and halo, respectively. This potential model was used to integrate the orbits of our candidates back half a billion years, and compute distributions of orbital properties including orbital period, orbital energy, apocenter, eccentricity, and angular momentum in z. All stars are shown to have orbits consistent with being disk stars.

\section{Discussion}  \label{sec:dis}

\subsection{Sodium enhancment}
According to \citet{el1995}, sodium overabundance observed in the atmospheres of supergiants is synthesized by the NeNa reaction chain in the convective core of main-sequence stars. Sodium is transported to the surface of these supergiants via mixing of CNO cycle products during the first dredge-up, although, it is important to note sodium enhancement is not constrained to supergiant stars. Stars with a  minimum 1.5$M_\odot$ can exhibit sodium abundance similarily through the NeNa cycle \citep{denissenkov1987}. \citet{antipova2004} discovered a sodium enhancement in 16 of their Barium star candidates, they relate this to the dredge-up of nuclear-burning material produced by convection during the red-giant phase. Moreover, they suggest [Na/Fe] ratios exhibit systematically higher values for giants with lower log g values, but do not exhibit variations in metallicity. Similarily, \citet{decastro2016} highlights the possible weak, anti-correlation between [Na/Fe] ratio and log g, and illustrates this trend in previous studies\citep[e.g.][]{boyarchuk2002,mishenina2006,luck2007,takeda2008}.

It's important to highlight our research only discovered $<2$\,\% sodium presence in our 257 Barium candidates, contrary to previous studies \citep[e.g.][]{decastro2016}. This suggests systematic differences in the analyse undertaken by said previous studies. Also, synonymous to our s-process candidates, synthesized sodium is brought to the surface of giants via the first dredge-up, therefore, these candidates may display sodium enhancement without mass transfer from a TP-AGB star. And, in juxtaposition to the literature, our 5 sodium enhanced stars exhibit a surface gravity range between 2.98 and 3, showing no correlation between higher [Na/Fe] values and lower log g values. These considerations suggest sodium-rich material from the TP-AGB star polluted our 5 Barium candidates, but also, sodium enhancement is not unbiquitous in Barium Stars. 

\subsection{Technetium enhancement}
Technetium has a half life of approximately 211,000 years and as a result cannot be present in an extrinsic Barium star. Enhancemnt in Tc would indicate the presence of AGB stars \citep{jorissen1993}, and suggest some of our candidates are intrinsic Barium stars. Our analyse discovered 51 single Tc line matches within our 257 Barium star candidate sample. However, for Tc to be present in  a star the enhancement lines must be considerably strong, and given that all 51 matches were at single lines it is concluded that these results are data artifacts. In agreement with previous findings \citep[e.g.][]{little1987,smith1984,smith1983}, the results indicate that none of the evolved Barium stars in our sample exhibited the presence of Tc enhancement, and it can be concluded our Barium star sample is extrinsic. It is noted, Tc enhancement is a very difficult line to analyse and that this conclusion is not concrete.

\subsection{Carbon bands}
Barium stars are excellent candidates for investigating the relationship between neutron-capture elements and other species that may be depleted or enhanced, since they act as neutron seeds during the operation of the s-process. In the AGB phase the abundance of carbon is a product of helium fusion, specifically the triple-alpha process within a star. In an identical mechanism to the presence of s-process element enhancement, carbon enhanced stars exist intrinsically or extrinsically. Giants and supergiants become enhanced in carbon through TDU process discussed in Section \ref{sec:intro}, and extrinsic carbon enhancement occurs via mass transfer from a binary AGB star. 

Our analsyse produced 178 CH and G band enhanced stars out of the total 257 Barium candidates. Similar to the s-process enhancement, this suggests our candidates are either intrinsic or extrinsic Barium stars.

\subsection{Extrinsic and intrinsic Barium stars}
Barium stars exist as either intrinsic or extrinsic objects. Intrinsic Barium stars must be massive enough to reach the TP-AGB phase, and extrinsic Barium stars must be in a binary with a previosuly polluting TP-AGB star companion. Figure \ref{fig:figure2} demonstrates the majority of our Barium star sample has a surface gravity greater then 2, highlighting that our sample is primarily not massive enough to have reach the TP-AGB phase and therefore, must be extrinsic Barium stars. Furthermore, this suggests other enhancements including our 5 sodium and 178 carbon enhanced Barium stars exist through TP-AGB companion pollution.

\begin{figure}
	% To include a figure from a file named example.*
	% Allowable file formats are eps or ps if compiling using latex
	% or pdf, png, jpg if compiling using pdflatex
	\includegraphics[width=\columnwidth]{HRdummy.png}
    \caption{effective temperature $T_\rm{eff}$ and surface gravity $\log{g}$ for 257 candidate Barium stars from LAMOST. Stars with $\log{g} > 2$ (as marked) have not evolved through the thermally pulsing asymptotic giant branch phase, and therefore their abundance signature must be the result of an extrinsic pollution event. The classification of stars with $\log{g} < 2$ would require additional data}
    \label{fig:figure2}
\end{figure}


\subsection{AGB yields comparison}

\section{Conclusions} \label{sec:con}

Our research discovered 967 s-process candidates including; 257 Barium stars, 49 stars exhibiting exclusively barium enhancement, and 661 exhibiting exclusively strontium enhancement from 454,180 giant stars observed by LAMOST. This is the largest sample of s-process enhanced stars to date. Nearly all of our Barium candidates are disk stars, 178 exhibit CH and G band enhancement, no plausible technetium enhancement is discovered, and it is shown sodium ehancement is not ubiquitous in s-process stars as previously claimed. We highlight that the majority of our Barium stars have a surface gravity greater then 2, this suggests that these are extrinsic Barium stars.

\section*{Acknowledgements}

Fired Status: yes

 
\bibliographystyle{mnras}
%\bibliography{example} % if your bibtex file is called example.bib

\begin{thebibliography}{99}
\bibitem[\protect\citeauthoryear{Allen \& Barbuy}{2006}]{allen2006}
Allen, D.~M.,\& Barbuy B. 2006, 
A$\&$A, 454, 917
\bibitem[\protect\citeauthoryear{Antipova et al.}{2004}]{antipova2004}
Antipova, L.~I., Boyarchuk, A.~A., Pakhomov, Yu.~V.,\& Panchuk, V.~E. 2004, 
ARep, 48, 597
\bibitem[\protect\citeauthoryear{Bidelman \& Keenan}{1951}]{Bidelman1951}
Bidelman, W.~P. \& Keenan, P.~C. , 1951, ApJ, 114, 473
\bibitem[\protect\citeauthoryear{Bisterzo et al.}{2014}]{bisterzo2014}
Bisterzo, S., et al. 2014, 
ApJ, 787, 10
A$\&$A, 424, 727
\bibitem[\protect\citeauthoryear{Boffin \& Jorissen}{1988}]{boffin1988}
Boffin H, M.~J.,\& Jorissen, A. 1988, 
A$\&$A, 205, 155
\bibitem[\protect\citeauthoryear{B\"ohm-Vitense}{1980}]{bohm1980}
B\"ohm-Vitense, E. 1980, 
ApJ, 239, L79
\bibitem[\protect\citeauthoryear{B\"ohm-Vitense et al.}{1984}]{bohm1984}
B\"ohm-Vitense, E., Nemec, J.,\& Proffitt, C. 1984, 
ApJ, 278, 726
\bibitem[\protect\citeauthoryear{Bovy}{2015}]{bovy2015}
Bovy, J. 2015, 
ApJ, 216, 29
\bibitem[\protect\citeauthoryear{Boyarchuk et al.}{2002}]{boyarchuk2002}
Boyarchuk, A.~A., Pakhomov, Y.~V., Antipova, L.~I.,\& Boyarchuk, M.~E. 2002, 
ARep, 46, 819
\bibitem[\protect\citeauthoryear{Busso et al.}{1999}]{busso1999}
Busso, M., Gallino, R., \& Wasserburg, P.~C. 1999, 
ARA$\&$A, 37, 239
\bibitem[\protect\citeauthoryear{Busso et al.}{2001}]{busso2001}
Busso, M., Gallino, R., Lambert, D.~L., Travaglio, C.\& Smith, V.~V. 2001, 
ApJ, 557, 802
\bibitem[\protect\citeauthoryear{deCastro et al.}{2016}]{decastro2016}
deCastro, D.~B., Pereira, C.~B., Roig, F., Jilinski, E., Drake, N.~A., Chavero, C.,\& Sales Silva, J.~V. 2016, 
MNRAS, 459, 4299
\bibitem[\protect\citeauthoryear{Denissenkov \& Ivanov}{1987}]{denissenkov1987}
Denissenkov P.~A.,\& Ivanov, V.~V. 1987, 
SvAL, 13, 214
\bibitem[\protect\citeauthoryear{Drake \& Pereira}{2008}]{drake2008}
Drake N.,~A.,\& Pereira, C.~B. 2008, 
AJ, 135, 1070
\bibitem[\protect\citeauthoryear{El Eid \& Champagne}{1995}]{el1995}
El Eid, M.~F.,\& Champagne, A.~E. 1995, 
ApJ, 451, 298
\bibitem[\protect\citeauthoryear{Gaia Collaboration et al.}{2016a}]{gaia2016a}
Gaia Collaboration, Brown, A.~G.~A., Vallenari, A., Prusti, T., de Bruijne, J.~H.~J., Mignard, F., Drimmel, R., Babusiaux, C., Bailer-Jones, C.~A.~L.,\& Bastian, U., et al. 2016, 
A$\&$A, 595, A2
\bibitem[\protect\citeauthoryear{Gaia Collaboration et al.}{2016b}]{gaia2016b}
Gaia Collaboration, Prusti, T., de Bruijne, J.~H.~J., Brown, A.~G.~A., Vallenari, A., Babusiaux, C., Bailer-Jones, C.~A.~L., Bastian, U., Biermann, M.,\& Evans, D.~W. et al. 2016, 
A$\&$A, 595, A1
\bibitem[\protect\citeauthoryear{Gomez et al.}{1997}]{gomez1997}
Gomez, A.~E., Luri, X., Grenier, S., et al. 1997, 
A$\&$A, 319, 881
\bibitem[\protect\citeauthoryear{Han et al.}{1995}]{han1995}
Han, Z., Eggleton, P.~P., Podsiadlowski, P.,\& Tout, C.~A. 1995, 
MNRAS, 277, 1443
\bibitem[\protect\citeauthoryear{Hernquist}{1990}]{hernquist1990}
Hernquist, L. 1990, 
ApJ, 356, 359
\bibitem[\protect\citeauthoryear{Herwig}{2005}]{herwig2005}
Herwig, F. 2005, 
ARA$\&$A, 43, 435
\bibitem[\protect\citeauthoryear{Ho et al.}{2017}]{ho2017}
Ho, A.~Y.~Q., Ness, M.~K., Hogg, D.~W., Rix, H.-W. Liu, C., Yang, F., Zhang, Y., Hou, Y.,\& Wang, Y. 2017, 
ApJ, 836, 5
\bibitem[\protect\citeauthoryear{Jorissen \& Boffin}{1992}]{jorissen1992}
Jorissen, A.,\& Boffin H, M.~J., 1992, 
Evidences for interaction among wide binary systems: To Ba or not to Ba? In: Duquennoy, A., Mayor, M.,(eds.) Binaries as tracers of stellar formation. Cambridge Univ. Press., p.185
\bibitem[\protect\citeauthoryear{Jorissen et al.}{1993}]{jorissen1993}
Jorissen, A., Frayer, D.~T., Johnson, H.~W., Mayor, M.,\& Smith, V.~V. 1993, 
A$\&$A, 271, 463
\bibitem[\protect\citeauthoryear{Jorissen et al.}{2005}]{jorissen2005}
Jorissen, A., Za$\check{c}$, L., Udry, S., Lindgren, H.,\& Musaev, F.~A. 2005, 
A$\&$A, 441, 1135
\bibitem[\protect\citeauthoryear{Junqueira \& Pereira}{2001}]{junqueira2001}
Junqueira S.,\& Pereira, C.~B. 2001, 
AJ, 122, 360
\bibitem[\protect\citeauthoryear{Karakas}{2016}]{karakas12016}
Karakas, A. 2016, 
SAIt, 87, 229
\bibitem[\protect\citeauthoryear{Kelson et al.}{2000}]{kelson2000}
Kelson, D.~D., Illingworth, G.~D., van Dokkum, P.~G.,\& Franx, M. 2000, ApJ, 531, 159
\bibitem[\protect\citeauthoryear{Kelson}{2003}]{kelson2003}
Kelson, D.~D. 2003, 
PASP, 115, 688
\bibitem[\protect\citeauthoryear{Kobayashi et al.}{2011}]{kobayashi2011}
Kobayashi, C., Karakas, A., \& Umeda, H. 2011, 
MNRAS, 414, 3231
\bibitem[\protect\citeauthoryear{Little-Marenin \& Little}{1987}]{little1987}
Little-Marenin, I.~R.,\& Little, S.~J. 1987, 
AJ, 93, 1539
\bibitem[\protect\citeauthoryear{Luck \& Heiter}{2007}]{luck2007}
Luck R.~E.,\& Heiter, U. 2007, 
AJ, 133, 2464
\bibitem[\protect\citeauthoryear{Luo et al}{2015}]{luo2015}
Luo, A.~L., Bai, Z.~R., et al. 2015, 
RAA, in press
\bibitem[\protect\citeauthoryear{McClure}{1983}]{mcclure1983}
McClure, R.~D. 1983, 
ApJ, 268, 264
\bibitem[\protect\citeauthoryear{Mennessier et al.}{1997}]{mennessier1997}
Mennessier, M.~O., Luri, X., Figueras, F., et al. 1997, 
A$\&$A, 326, 722
\bibitem[\protect\citeauthoryear{Mishenina et al.}{2006}]{mishenina2006}
Mishenina, T.~V., Bienaym\' e, O., Gorbaneva, T.~I.,\& Charbonnel, C. 2006, 
A$\&$A, 456, 1109
\bibitem[\protect\citeauthoryear{Miyamoto-Nagai}{1975}]{miyamoto1975}
Miyamoto, M,\& Nagai, R. 1975, 
PASJ, 27, 533
\bibitem[\protect\citeauthoryear{NFW}{1997}]{nfw1997}
Navarro, J.~F., Frenk, C.~S.,\& White, S.~D.~M. 1997, 
ApJ, 490, 493
\bibitem[\protect\citeauthoryear{Pereira \& Drake}{2009}]{pereira2009}
Pereira, C.~B.,\& Drake N.,~A. 2009, 
A$\&$A, 496, 791
\bibitem[\protect\citeauthoryear{Pereira et al.}{2011}]{pereira2011}
Pereira, C.~B., Sales Silva, J.,~A., Chavero, C., Roig, F.,\& Jilinski E. 2011, 
A$\&$A, 533, A51
\bibitem[\protect\citeauthoryear{Pourbaix et al.}{2004}]{pourbaix2004}
Pourbaix, D., Tokovinin, A.~A., Batten, A.~H., Fekel, F.~C., Hartkopf, W.~I. et al. 2001, 
\bibitem[\protect\citeauthoryear{Prantzos}{2012}]{prantzos2012}
Prantzos, N. 2012, 
A$\&$A, 542, A67
\bibitem[\protect\citeauthoryear{Price-Whelan}{2017}]{price2017}
Price-Whelan, A.~M. 2017, 
The Journal of Open Source Software, 2, 388
\bibitem[\protect\citeauthoryear{Romano et al.}{2010}]{romano2010}
Romano, D., et al. 2010, 
A$\&$A, 522, A32
\bibitem[\protect\citeauthoryear{Smith \& Wallerstein}{1983}]{smith1983}
Smith, V.~V.,\& Wallerstein, G. 1983, 
ApJ, 273, 742
\bibitem[\protect\citeauthoryear{Smith}{1984}]{smith1984}
Smith, V.~V. 1984, 
A$\&$A, 132, 326
\bibitem[\protect\citeauthoryear{Takeda et al.}{2008}]{takeda2008}
Takeda, Y., Sato, B.~V.,\& Murata, D. 2008, 
AJ, 60, 781
\bibitem[\protect\citeauthoryear{Travaglio et al.}{2001}]{travaglio2001}
Travaglio, C., et al. 2001, 
ApJ, 549, 346
\bibitem[\protect\citeauthoryear{Webbink}{1986}]{webbink1986}
Webbink, R.~F. 1986, 
In: Leung, K.~C., Zhai, D.~S.(eds.) Critical Observations versus Physical Models for Close Binary Systems. Gordon and Breach, New York, p.403
\bibitem[\protect\citeauthoryear{Whitelock et al.}{2013}]{whitelock2013}
Whitelock, P.~A., et al. 2013, 
MNRAS, 428, 2216

\bibitem[\protect\citeauthoryear{Casey}{2014}]{casey2014}
Casey, A.~R. 2014, 
PhDT, 394
\todo{format Andy's ref properly}
%@PHDTHESIS{2014PhDT.......394C,
%   author = {{Casey}, A.~R.},
%    title = "{A Tale of Tidal Tales in the Milky Way}",
% keywords = {Astronomy, Astrophysics, Stellar streams},
%   school = {Australian National University},
%     year = 2014,
%    month = May,
%      doi = {10.5281/zenodo.49493},
%   adsurl = {http://adsabs.harvard.edu/abs/2014PhDT.......394C},
%  adsnote = {Provided by the SAO/NASA Astrophysics Data System}
%}

\end{thebibliography}


% Don't change these lines
\bsp	% typesetting comment
\label{lastpage}
\end{document}